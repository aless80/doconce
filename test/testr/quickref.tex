%%
%% Automatically generated file from DocOnce source
%% (https://github.com/doconce/doconce/)
%%
% #define PREAMBLE
% #ifdef PREAMBLE
%-------------------- begin preamble ----------------------
\documentclass[%
oneside,                 % oneside: electronic viewing, twoside: printing
final,                   % draft: marks overfull hboxes, figures with paths
10pt]{article}
\listfiles               %  print all files needed to compile this document
\usepackage{relsize,makeidx,color,setspace,amsmath,amsfonts,amssymb}
\usepackage[table]{xcolor}
\usepackage{bm,ltablex,microtype}
\usepackage[pdftex]{graphicx}
% Tools for marking corrections
\usepackage{soul}
\newcommand{\replace}[2]{{\color{red}\text{\st{#1} #2}}}
\newcommand{\remove}[1]{{\color{red}\st{#1}}}
% Movies are handled by the href package
\newenvironment{doconce:movie}{}{}
\newcounter{doconce:movie:counter}
% Packages for typesetting blocks of computer code
\usepackage{fancyvrb,framed,moreverb}
% Define colors
\definecolor{orange}{cmyk}{0,0.4,0.8,0.2}
\definecolor{tucorange}{rgb}{1.0,0.64,0}
\definecolor{darkorange}{rgb}{.71,0.21,0.01}
\definecolor{darkgreen}{rgb}{.12,.54,.11}
\definecolor{myteal}{rgb}{.26, .44, .56}
\definecolor{gray}{gray}{0.45}
\definecolor{mediumgray}{gray}{.8}
\definecolor{lightgray}{gray}{.95}
\definecolor{brown}{rgb}{0.54,0.27,0.07}
\definecolor{purple}{rgb}{0.5,0.0,0.5}
\definecolor{darkgray}{gray}{0.25}
\definecolor{darkblue}{rgb}{0,0.08,0.45}
\definecolor{darkblue2}{rgb}{0,0,0.8}
\definecolor{lightred}{rgb}{1.0,0.39,0.28}
\definecolor{lightgreen}{rgb}{0.48,0.99,0.0}
\definecolor{lightblue}{rgb}{0.53,0.81,0.92}
\definecolor{lightblue2}{rgb}{0.3,0.3,1.0}
\definecolor{lightpurple}{rgb}{0.87,0.63,0.87}
\definecolor{lightcyan}{rgb}{0.5,1.0,0.83}
\colorlet{comment_green}{green!50!black}
\colorlet{string_red}{red!60!black}
\colorlet{keyword_pink}{magenta!70!black}
\colorlet{indendifier_green}{green!70!white}
% Backgrounds for code
\definecolor{cbg_gray}{rgb}{.95, .95, .95}
\definecolor{bar_gray}{rgb}{.92, .92, .92}
\definecolor{cbg_yellowgray}{rgb}{.95, .95, .85}
\definecolor{bar_yellowgray}{rgb}{.95, .95, .65}
\colorlet{cbg_yellow2}{yellow!10}
\colorlet{bar_yellow2}{yellow!20}
\definecolor{cbg_yellow1}{rgb}{.98, .98, 0.8}
\definecolor{bar_yellow1}{rgb}{.98, .98, 0.4}
\definecolor{cbg_red1}{rgb}{1, 0.85, 0.85}
\definecolor{bar_red1}{rgb}{1, 0.75, 0.85}
\definecolor{cbg_blue1}{rgb}{0.87843, 0.95686, 1.0}
\definecolor{bar_blue1}{rgb}{0.7,     0.95686, 1}
\usepackage[T1]{fontenc}
%\usepackage[latin1]{inputenc}
\usepackage{ucs}
\usepackage[utf8x]{inputenc}
% Set helvetica as the default font family:
\RequirePackage{helvet}
\renewcommand\familydefault{phv}
\usepackage{lmodern}         % Latin Modern fonts derived from Computer Modern
% Hyperlinks in PDF:
\definecolor{linkcolor}{rgb}{0,0,0.4}
\usepackage{hyperref}
\hypersetup{
    breaklinks=true,
    colorlinks=true,
    linkcolor=linkcolor,
    urlcolor=linkcolor,
    citecolor=black,
    filecolor=black,
    %filecolor=blue,
    pdfmenubar=true,
    pdftoolbar=true,
    bookmarksdepth=3   % Uncomment (and tweak) for PDF bookmarks with more levels than the TOC
    }
%\hyperbaseurl{}   % hyperlinks are relative to this root
\setcounter{tocdepth}{2}  % levels in table of contents
% Tricks for having figures close to where they are defined:
% 1. define less restrictive rules for where to put figures
\setcounter{topnumber}{2}
\setcounter{bottomnumber}{2}
\setcounter{totalnumber}{4}
\renewcommand{\topfraction}{0.95}
\renewcommand{\bottomfraction}{0.95}
\renewcommand{\textfraction}{0}
\renewcommand{\floatpagefraction}{0.75}
% floatpagefraction must always be less than topfraction!
% 2. ensure all figures are flushed before next section
\usepackage[section]{placeins}
% 3. enable begin{figure}[H] (often leads to ugly pagebreaks)
%\usepackage{float}\restylefloat{figure}
% newcommands for typesetting inline (doconce) comments
\newcommand{\shortinlinecomment}[3]{{\color{red}{\bf #1}: #2}}
\newcommand{\longinlinecomment}[3]{{\color{red}{\bf #1}: #2}}
\usepackage[framemethod=TikZ]{mdframed}
% --- begin definitions of admonition environments ---
% Admonition style "mdfbox" is an oval colored box based on mdframed
% "notice" admon
\colorlet{mdfbox_notice_background}{gray!5}
\newmdenv[
  skipabove=15pt,
  skipbelow=15pt,
  outerlinewidth=0,
  backgroundcolor=mdfbox_notice_background,
  linecolor=black,
  linewidth=2pt,       % frame thickness
  frametitlebackgroundcolor=mdfbox_notice_background,
  frametitlerule=true,
  frametitlefont=\normalfont\bfseries,
  shadow=false,        % frame shadow?
  shadowsize=11pt,
  leftmargin=0,
  rightmargin=0,
  roundcorner=5,
  needspace=0pt,
]{notice_mdfboxmdframed}
\newenvironment{notice_mdfboxadmon}[1][]{
\begin{notice_mdfboxmdframed}[frametitle=#1]
}
{
\end{notice_mdfboxmdframed}
}
% Admonition style "mdfbox" is an oval colored box based on mdframed
% "summary" admon
\colorlet{mdfbox_summary_background}{gray!5}
\newmdenv[
  skipabove=15pt,
  skipbelow=15pt,
  outerlinewidth=0,
  backgroundcolor=mdfbox_summary_background,
  linecolor=black,
  linewidth=2pt,       % frame thickness
  frametitlebackgroundcolor=mdfbox_summary_background,
  frametitlerule=true,
  frametitlefont=\normalfont\bfseries,
  shadow=false,        % frame shadow?
  shadowsize=11pt,
  leftmargin=0,
  rightmargin=0,
  roundcorner=5,
  needspace=0pt,
]{summary_mdfboxmdframed}
\newenvironment{summary_mdfboxadmon}[1][]{
\begin{summary_mdfboxmdframed}[frametitle=#1]
}
{
\end{summary_mdfboxmdframed}
}
% Admonition style "mdfbox" is an oval colored box based on mdframed
% "warning" admon
\colorlet{mdfbox_warning_background}{gray!5}
\newmdenv[
  skipabove=15pt,
  skipbelow=15pt,
  outerlinewidth=0,
  backgroundcolor=mdfbox_warning_background,
  linecolor=black,
  linewidth=2pt,       % frame thickness
  frametitlebackgroundcolor=mdfbox_warning_background,
  frametitlerule=true,
  frametitlefont=\normalfont\bfseries,
  shadow=false,        % frame shadow?
  shadowsize=11pt,
  leftmargin=0,
  rightmargin=0,
  roundcorner=5,
  needspace=0pt,
]{warning_mdfboxmdframed}
\newenvironment{warning_mdfboxadmon}[1][]{
\begin{warning_mdfboxmdframed}[frametitle=#1]
}
{
\end{warning_mdfboxmdframed}
}
% Admonition style "mdfbox" is an oval colored box based on mdframed
% "question" admon
\colorlet{mdfbox_question_background}{gray!5}
\newmdenv[
  skipabove=15pt,
  skipbelow=15pt,
  outerlinewidth=0,
  backgroundcolor=mdfbox_question_background,
  linecolor=black,
  linewidth=2pt,       % frame thickness
  frametitlebackgroundcolor=mdfbox_question_background,
  frametitlerule=true,
  frametitlefont=\normalfont\bfseries,
  shadow=false,        % frame shadow?
  shadowsize=11pt,
  leftmargin=0,
  rightmargin=0,
  roundcorner=5,
  needspace=0pt,
]{question_mdfboxmdframed}
\newenvironment{question_mdfboxadmon}[1][]{
\begin{question_mdfboxmdframed}[frametitle=#1]
}
{
\end{question_mdfboxmdframed}
}
% Admonition style "mdfbox" is an oval colored box based on mdframed
% "block" admon
\colorlet{mdfbox_block_background}{gray!5}
\newmdenv[
  skipabove=15pt,
  skipbelow=15pt,
  outerlinewidth=0,
  backgroundcolor=mdfbox_block_background,
  linecolor=black,
  linewidth=2pt,       % frame thickness
  frametitlebackgroundcolor=mdfbox_block_background,
  frametitlerule=true,
  frametitlefont=\normalfont\bfseries,
  shadow=false,        % frame shadow?
  shadowsize=11pt,
  leftmargin=0,
  rightmargin=0,
  roundcorner=5,
  needspace=0pt,
]{block_mdfboxmdframed}
\newenvironment{block_mdfboxadmon}[1][]{
\begin{block_mdfboxmdframed}[frametitle=#1]
}
{
\end{block_mdfboxmdframed}
}
% --- end of definitions of admonition environments ---
% prevent orhpans and widows
\clubpenalty = 10000
\widowpenalty = 10000
\usepackage{calc}
\newenvironment{doconceexercise}{}{}
\newcounter{doconceexercisecounter}
% ------ header in subexercises ------
%\newcommand{\subex}[1]{\paragraph{#1}}
%\newcommand{\subex}[1]{\par\vspace{1.7mm}\noindent{\bf #1}\ \ }
\makeatletter
% 1.5ex is the spacing above the header, 0.5em the spacing after subex title
\newcommand\subex{\@startsection{paragraph}{4}{\z@}%
                  {1.5ex\@plus1ex \@minus.2ex}%
                  {-0.5em}%
                  {\normalfont\normalsize\bfseries}}
\makeatother
% --- end of standard preamble for documents ---
%%% USER-DEFINED ENVIRONMENTS
% insert custom LaTeX commands...
\raggedbottom
\makeindex
\usepackage[totoc]{idxlayout}   % for index in the toc
\usepackage[nottoc]{tocbibind}  % for references/bibliography in the toc
%-------------------- end preamble ----------------------
\begin{document}
% matching end for #ifdef PREAMBLE
% #endif
\newcommand{\exercisesection}[1]{\subsection*{#1}}
% ------------------- main content ----------------------
% ----------------- title -------------------------
\thispagestyle{empty}
\begin{center}
{\LARGE\bf
\begin{spacing}{1.25}
DocOnce Quick Reference
\end{spacing}
}
\end{center}
% ----------------- author(s) -------------------------
\begin{center}
{\bf Hans Petter Langtangen${}^{1, 2}$} \\ [0mm]
\end{center}
\begin{center}
% List of all institutions:
\centerline{{\small ${}^1$Center for Biomedical Computing, Simula Research Laboratory}}
\centerline{{\small ${}^2$Department of Informatics, University of Oslo}}
\end{center}
    
% ----------------- end author(s) -------------------------
% --- begin date ---
\begin{center}
Jan 32, 2100
\end{center}
% --- end date ---
\vspace{1cm}
\tableofcontents
\vspace{1cm} % after toc
\textbf{WARNING: This quick reference is very incomplete!}
\paragraph{Mission.}
Enable writing documentation with much mathematics and
computer code \emph{once, in one place} and include it in traditional {\LaTeX}
books, thesis, and reports, and, without extra efforts, also make
professional looking web versions with Sphinx or HTML. Other outlets
include Google's \texttt{blogger.com}, Wikipedia/Wikibooks, IPython/Jupyter
notebooks, plus a wide variety of other formats for documents without
mathematics and code.
\subsection{Supported Formats}
DocOnce currently translates files to the following formats:
\begin{itemize}
 \item {\LaTeX} (format \texttt{latex} or \texttt{pdflatex})
 \item HTML (format \texttt{html})
 \item Sphinx (format \texttt{sphinx})
 \item Pandoc-extended or GitHub-flavored Markdown (format \texttt{pandoc})
 \item IPython notebook (format \texttt{ipynb})
 \item Matlab notebook (format \texttt{matlabnb})
 \item MediaWiki (format \texttt{mwiki})
 \item Googlecode wiki (format \texttt{gwiki})
 \item Creoloe wiki (format \texttt{cwiki})
 \item reStructuredText (format \texttt{rst})
 \item plain (untagged) ASCII (format \texttt{plain})
 \item Epydoc (format \texttt{epydoc})
 \item StructuredText (format \texttt{st})
\end{itemize}
\noindent
For documents with much code and mathematics, the best (and most supported)
formats are \texttt{latex}, \texttt{pdflatex}, \texttt{sphinx}, and \texttt{html}; and to a slightly
less extent \texttt{mwiki} and \texttt{pandoc}. The HTML format supports blog posts on
Google and Wordpress.

\begin{warning_mdfboxadmon}[Use a text editor with monospace font!]
Some DocOnce constructions are sensitive to whitespace,
so you \emph{must} use a text editor with monospace font.
\end{warning_mdfboxadmon} % title: Use a text editor with monospace font!


\subsection{Emacs syntax support}
The file \href{{https://github.com/doconce/doconce/blob/master/misc/.doconce-mode.el}}{.doconce-mode.el}
in the DocOnce source distribution gives a "DocOnce Editing Mode" in
Emacs. Store the raw version of the file in the home directory and add
\texttt{(load-file "~/.doconce-mode.el")} to the \texttt{.emacs} file.
Besides syntax highlighting of DocOnce documents, this Emacs mode
provides a lot of shortcuts for setting up many elements in a document:
\begin{quote}
\begin{tabular}{ll}
\hline
\multicolumn{1}{c}{ Emacs key } & \multicolumn{1}{c}{ Action } \\
\hline
Ctrl+c f      & figure                                           \\
Ctrl+c v      & movie/video                                      \\
Ctrl+c h1     & heading level 1 (section/h1)                     \\
Ctrl+c h2     & heading level 2 (subsection/h2)                  \\
Ctrl+c h3     & heading level 2 (subsection/h3)                  \\
Ctrl+c hp     & heading for paragraph                            \\
Ctrl+c me     & math environment: \Verb?!bt? equation \Verb?!et? \\
Ctrl+c ma     & math environment: \Verb?!bt? align \Verb?!et?    \\
Ctrl+c ce     & code environment: \Verb?!bc? code \Verb?!ec?     \\
Ctrl+c cf     & code from file: \texttt{@@@CODE}                   \\
Ctrl+c table2 & table with 2 columns                             \\
Ctrl+c table3 & table with 3 columns                             \\
Ctrl+c table4 & table with 4 columns                             \\
Ctrl+c exer   & exercise outline                                 \\
Ctrl+c slide  & slide outline                                    \\
Ctrl+c help   & print this table                                 \\
\hline
\end{tabular}
\end{quote}
\noindent
\subsection{Title, Authors, and Date}
A typical example of giving a title, a set of authors, a date,
and an optional table of contents
reads
\begin{Verbatim}[numbers=none,fontsize=\fontsize{9pt}{9pt},baselinestretch=0.95,xleftmargin=2mm]
TITLE: On an Ultimate Markup Language
AUTHOR: H. P. Langtangen at Center for Biomedical Computing, Simula Research Laboratory & Dept. of Informatics, Univ. of Oslo
AUTHOR: Kaare Dump Email: dump@cyb.space.com at Segfault, Cyberspace Inc.
AUTHOR: A. Dummy Author
DATE: today
TOC: on

\end{Verbatim}

The entire title must appear on a single line.
The author syntax is
\begin{Verbatim}[numbers=none,fontsize=\fontsize{9pt}{9pt},baselinestretch=0.95,xleftmargin=2mm]
name Email: somename@adr.net at institution1 & institution2

\end{Verbatim}

where the email is optional, the "at" keyword is required if one or
more institutions are to be specified, and the \Verb!&! keyword
separates the institutions (the keyword \texttt{and} works too).
Each author specification must appear
on a single line.
When more than one author belong to the
same institution, make sure that the institution is specified in an identical
way for each author.
The date can be set as any text different from \texttt{today} if not the
current date is wanted, e.g., \texttt{Jan 32, 2100}.
The table of contents is removed by writing \texttt{TOC: off}.
\subsection{Copyright}
% Recall to ident AUTHOR commands to avoid interpretation
Copyright for selected authors and/or institutions are easy to insert as part
of the \texttt{AUTHOR} command. The copyright syntax is
\begin{Verbatim}[numbers=none,fontsize=\fontsize{9pt}{9pt},baselinestretch=0.95,xleftmargin=2mm]
{copyright,year1-year2|license}

\end{Verbatim}

and can be placed after the author or after an institution, e.g.,
\begin{Verbatim}[numbers=none,fontsize=\fontsize{9pt}{9pt},baselinestretch=0.95,xleftmargin=2mm]
 AUTHOR: name Email: somename@adr.net {copyright,2006-present} at inst1
 AUTHOR: name {copyright} at inst1 {copyright}

\end{Verbatim}

The first line gives \texttt{name} a copyright for 2006 up to the present year,
while the second line gives copyright to \texttt{name} and the institution \texttt{inst1}
for the present year. The license can be any formulation, but there are
some convenient abbreviations for Creative Commons (``public domain'')
licenses: \texttt{CC BY} for Creative Commons Attribution 4.0 license,
\texttt{CC BY-NC} for Creative Commons Attribution-NonCommercial 4.0 license.
For example,
\begin{Verbatim}[numbers=none,fontsize=\fontsize{9pt}{9pt},baselinestretch=0.95,xleftmargin=2mm]
 AUTHOR: name1 {copyright|CC BY} at institution1
 AUTHOR: name2 {copyright|CC BY} at institution2

\end{Verbatim}

is a very common copyright for the present year with the Attribution license.
The copyright must be identical for all authors and institutions.
\subsection{Section Types}
\label{quick:sections}
\begin{quote}
\begin{tabular}{ll}
\hline
\multicolumn{1}{c}{ Section type } & \multicolumn{1}{c}{ Syntax } \\
\hline
chapter       & \texttt{========= Heading ========} (9 \texttt{=})        \\
section       & \texttt{======= Heading =======}    (7 \texttt{=})        \\
subsection    & \texttt{===== Heading =====}        (5 \texttt{=})        \\
subsubsection & \texttt{=== Heading ===}            (3 \texttt{=})        \\
paragraph     & \Verb!__Heading.__!               (2 \Verb!_!)        \\
abstract      & \Verb!__Abstract.__! Running text...                  \\
appendix      & \texttt{======= Appendix: heading =======} (7 \texttt{=}) \\
appendix      & \texttt{===== Appendix: heading =====} (5 \texttt{=})     \\
exercise      & \texttt{===== Exercise: heading =====} (5 \texttt{=})     \\
\hline
\end{tabular}
\end{quote}
\noindent
Note that abstracts are recognized by starting with \Verb!__Abstract.__! or
\Verb!__Summary.__! at the beginning of a line and ending with three or
more \texttt{=} signs of the next heading.
The \texttt{Exercise:} keyword can be substituted by \texttt{Problem:} or \texttt{Project:}.
A recommended convention is that an exercise is tied to the text,
a problem can stand on its own, and a project is a comprehensive
problem.
\subsection{Inline Formatting}
Words surrounded by \texttt{*} are emphasized: \texttt{*emphasized words*} becomes
\emph{emphasized words}. Similarly, an underscore surrounds words that
appear in boldface: \Verb!_boldface_! becomes \textbf{boldface}. Colored words
are also possible: the text
\begin{Verbatim}[numbers=none,fontsize=\fontsize{9pt}{9pt},baselinestretch=0.95,xleftmargin=2mm]
`color{red}{two red words}`

\end{Verbatim}

becomes \textcolor{red}{two red words}.
Quotations appear inside double backticks and double single quotes:
\begin{Verbatim}[numbers=none,fontsize=\fontsize{9pt}{9pt},baselinestretch=0.95,xleftmargin=2mm]
This is a sentence with ``words to be quoted''.

\end{Verbatim}

A forced linebreak is specified by \texttt{<linebreak>} at the point where the
linebreak in the output is wanted.
Footnotes use a label in the text with the footnote text separate,
preferably after the paragraph where the footnote appears:
\begin{Verbatim}[numbers=none,fontsize=\fontsize{9pt}{9pt},baselinestretch=0.95,xleftmargin=2mm]
Differentiating[^diff2] this equation leads
to a new and much simpler equation.

[^diff2]: More precisely, we apply the divergence
$\nabla\cdot$ on both sides.

Here comes a new paragraph...

\end{Verbatim}

Non-breaking space is inserted using the tilde character as in {\LaTeX}:
\begin{Verbatim}[numbers=none,fontsize=\fontsize{9pt}{9pt},baselinestretch=0.95,xleftmargin=2mm]
This distance corresponds to 7.5~km, which is traveled in $7.5/5$~s.

\end{Verbatim}

A horizontal rule for separating content vertically, like this:
-----
is typeset as four or more hyphens on a single line:
\begin{Verbatim}[numbers=none,fontsize=\fontsize{9pt}{9pt},baselinestretch=0.95,xleftmargin=2mm]
---------

\end{Verbatim}

The \texttt{latex}, \texttt{pdflatex}, \texttt{sphinx}, and \texttt{html} formats support em-dash,
indicated by three hyphens: \texttt{---}. Here is an example:
\begin{Verbatim}[numbers=none,fontsize=\fontsize{9pt}{9pt},baselinestretch=0.95,xleftmargin=2mm]
The em-dash is used - without spaces - as alternative to hyphen with
space around in sentences---this way, or in quotes:
*Premature optimization is the root of all evil.*--- Donald Knuth.

\end{Verbatim}

This text is in the pdflatex rendered as
The em-dash is used - without spaces - as alternative to hyphen with
space around in sentences---this way, or in quotes:
\emph{Premature optimization is the root of all evil.}--- Donald Knuth.
The en-dash consists of two hyphens, either with blanks on both sides -- for
something in the middle of a sentence -- or in number ranges like 240--249.
{\LaTeX} writes are used to and fond of en-dash.
An ampersand, as in Guns {\&} Roses or,Texas A {\&} M, is written as a
plain \Verb!&! \emph{with space(s) on both sides}. Single upper case letters on each
side of \Verb!&!, as in \Verb!Texas A {\&} M!, remove the spaces and result in,Texas A {\&} M, while words on both sides of \Verb!&!, as in \Verb!Guns {\&} Roses!,
preserve the spaces: Guns {\&} Roses. Failing to have spaces before and
after \Verb!&! will result in wrong typesetting of the ampersand in the \texttt{html},
\texttt{latex}, and \texttt{pdflatex} formats.
Emojis, as defined in \href{{http://www.emoji-cheat-sheet.com}}{\nolinkurl{http://www.emoji-cheat-sheet.com}}, can be
inserted in the text, as (e.g.) \Verb!:dizzy_face:! with blank or newline
before or after \raisebox{-\height+\ht\strutbox}{\includegraphics[height=1.5em]{latex_figs/dizzy_face.png}} Only the \texttt{pdflatex}, \texttt{html}, and \texttt{pandoc} output
formats translate emoji specifications to images, while all other
formats leave the textual specification in the document. The
command-line option \Verb!--no_emoji! removes all emojis from the output
document.
\subsection{Lists}
There are three types of lists: \emph{bullet lists}, where each item starts
with \texttt{*}, \emph{enumeration lists}, where each item starts with \texttt{o} and gets
consecutive numbers,
and \emph{description} lists, where each item starts with \texttt{-} followed
by a keyword and a colon.
\begin{Verbatim}[numbers=none,fontsize=\fontsize{9pt}{9pt},baselinestretch=0.95,xleftmargin=2mm]
Here is a bullet list:

 * item1
 * item2
  * subitem1 of item2
  * subitem2 of item2,
    second line of subitem2
 * item3

Note that sublists are consistently indented by one or more blanks as
shown: bullets must exactly match and continuation lines must start
right below the line above.

Here is an enumeration list:

 o item1
 o item2
   may appear on
   multiple lines
  o subitem1 of item2
  o subitem2 of item2
 o item3

And finally a description list:

 - keyword1: followed by
   some text
   over multiple
   lines
 - keyword2:
   followed by text on the next line
 - keyword3: and its description may fit on one line

\end{Verbatim}

The code above follows.
Here is a bullet list:
\begin{itemize}
 \item item1
 \item item2
\begin{itemize}
  \item subitem1 of item2
  \item subitem2 of item2
\end{itemize}
\noindent
 \item item3
\end{itemize}
\noindent
Note that sublists are consistently indented by one or more blanks as
shown: bullets must exactly match and continuation lines must start
right below the line above.
Here is an enumeration list:
\begin{enumerate}
\item item1
\item item2
   may appear on
   multiple lines
\begin{enumerate}
 \item subitem1 of item2
 \item subitem2 of item2
\end{enumerate}
\noindent
\item item3
\end{enumerate}
\noindent
And finally a description list:
\begin{description}
 \item[keyword1:] 
   followed by
   some text
   over multiple
   lines
 \item[keyword2:] 
   followed by text on the next line
 \item[keyword3:] 
   and its description may fit on one line
\end{description}
\noindent

\begin{warning_mdfboxadmon}[No indentation - except in lists!]
DocOnce syntax is sensitive to whitespace.
No lines should be indented, only lines belonging to lists.
Indented lines may give strange output in some formats.
\end{warning_mdfboxadmon} % title: No indentation - except in lists!


\subsection{Comment lines}
Lines starting with \Verb!#! are treated as comments in the document and
translated to the proper syntax for comments in the output
document. Such comment lines should not appear before {\LaTeX} math
blocks, verbatim code blocks, or lists if the formats \texttt{rst} and
\texttt{sphinx} are desired.
Comment lines starting with \Verb!##! are not propagated to the output
document and can be used for comments that are only of interest in
the DocOnce file.
Large portions of text can be left out using Preprocess. Just place
\Verb!# #ifdef EXTRA! and \Verb!# #endif! around the text. The command line
option \texttt{-DEXTRA} will bring the text alive again.
When using the Mako preprocessor one can also place comments in
the DocOnce source file that will be removed by Mako before
DocOnce starts processing the file.
\subsection{Inline comments}
Inline comments meant as messages or notes, to authors during development
in particular,
are enabled by the syntax
\begin{Verbatim}[numbers=none,fontsize=\fontsize{9pt}{9pt},baselinestretch=0.95,xleftmargin=2mm]
[name: running text]

\end{Verbatim}

where \texttt{name} is the name or ID of an author or reader making the comment,
and \texttt{running text} is the comment. The name can contain upper and lower
case characters, digits, single quote, \texttt{+} and \texttt{-}, as well
as space. Here goes an example.
\begin{Verbatim}[numbers=none,fontsize=\fontsize{9pt}{9pt},baselinestretch=0.95,xleftmargin=2mm]
Some running text. [hpl: There must be a space after the colon,
but the running text can occupy multiple lines.]

\end{Verbatim}

which is rendered as

\begin{quote}
Some running text. \shortinlinecomment{hpl 1}{ There must be a space after the colon, but the running text can occupy multiple lines. }{ There must be a }
\end{quote}

The inline comments have simple typesetting in most formats, typically
boldface name, a comment number, with everything surrounded by
parenthesis.  However, with {\LaTeX} output and the \Verb!--latex_todonotes!
option to \texttt{doconce format}, colorful margin or inline boxes (using the
\texttt{todonotes} package) make it very easy to spot the comments.
Running
\begin{Verbatim}[numbers=none,fontsize=\fontsize{9pt}{9pt},baselinestretch=0.95,xleftmargin=2mm]
doconce format html mydoc.do.txt --skip_inline_comments

\end{Verbatim}

removes all inline comments from the output. This feature makes it easy
to turn on and off notes to authors during the development of the document.
All inline comments to readers can also be physically
removed from the DocOnce source by
\begin{Verbatim}[numbers=none,fontsize=\fontsize{9pt}{9pt},baselinestretch=0.95,xleftmargin=2mm]
doconce remove_inline_comments mydoc.do.txt

\end{Verbatim}

Inline comments can also be used to markup edits. There are add, delete, and
replacement comments for editing:
\begin{Verbatim}[numbers=none,fontsize=\fontsize{9pt}{9pt},baselinestretch=0.95,xleftmargin=2mm]
[add: ,]
[add: .]
[add: ;]
[del: ,]
[del: ,]
[del: .]
[del: ;]
[add: some text]
[del: some text]
[edit: some text -> some replacement for text]
[name: some text -> some replacement for text]

\end{Verbatim}

For example, the text
\begin{Verbatim}[numbers=none,fontsize=\fontsize{9pt}{9pt},baselinestretch=0.95,xleftmargin=2mm]
First consider a quantity $Q$. Without loss of generality, we assume
$Q>0$. There are three, fundamental, basic property of $Q$.

\end{Verbatim}

can be edited as
\begin{Verbatim}[numbers=none,fontsize=\fontsize{9pt}{9pt},baselinestretch=0.95,xleftmargin=2mm]
First[add: ,] consider [edit: a quantity -> the flux]
[del: $Q$. Without loss of generality,
we assume] $Q>0$. There are three[del: ,] fundamental[del: , basic]
[edit: property -> properties] of $Q$. [add: These are not
important for the following discussion.]

\end{Verbatim}

which in the pdflatex output format results in

\begin{quote}
First\textcolor{red}{, (\textbf{edit 2}: add comma)} consider \textcolor{red}{(edit 3:)} \replace{a quantity}{the flux}
(\textbf{edit 4}:) \remove{$Q$. Without loss of generality,
we assume} $Q>0$. There are three \textcolor{red}{ (\textbf{edit 5}: delete comma)} fundamental(\textbf{edit 6}:) \remove{, basic}
\textcolor{red}{(edit 7:)} \replace{property}{properties} of $Q$.  \textcolor{red}{ (\textbf{edit 8}:) These are not
important for the following discussion.}
\end{quote}

To implement these edits, run
\begin{Verbatim}[numbers=none,fontsize=\fontsize{9pt}{9pt},baselinestretch=0.95,xleftmargin=2mm]
Terminal> doconce apply_edit_comments mydoc.do.txt

\end{Verbatim}

\subsection{Verbatim/Computer Code}
Inline verbatim code is typeset within back-ticks, as in
\begin{Verbatim}[numbers=none,fontsize=\fontsize{9pt}{9pt},baselinestretch=0.95,xleftmargin=2mm]
Some sentence with `words in verbatim style`.

\end{Verbatim}

resulting in Some sentence with \texttt{words in verbatim style}.
Multi-line blocks of verbatim text, typically computer code, is typeset
in between \Verb?!bc xxx? and \Verb?!ec? directives, which must appear on the
beginning of the line. A specification \texttt{xxx} indicates what verbatim
formatting style that is to be used. Typical values for \texttt{xxx} are
nothing, \texttt{cod} for a code snippet, \texttt{pro} for a complete program,
\texttt{sys} for a terminal session, \texttt{dat} for a data file (or output from a
program),
\texttt{Xpro} or \texttt{Xcod} for a program or code snipped, respectively,
in programming \texttt{X}, where \texttt{X} may be \texttt{py} for Python,
\texttt{cy} for Cython, \texttt{sh} for Bash or other Unix shells,
\texttt{f} for Fortran, \texttt{c} for C, \texttt{cpp} for C++, \texttt{m} for MATLAB,
\texttt{pl} for Perl. For output in \texttt{latex} one can let \texttt{xxx} reflect any
defined verbatim environment in the \texttt{ptex2tex} configuration file
(\texttt{.ptex2tex.cfg}). For \texttt{sphinx} output one can insert a comment
\begin{Verbatim}[numbers=none,fontsize=\fontsize{9pt}{9pt},baselinestretch=0.95,xleftmargin=2mm]
# sphinx code-blocks: pycod=python cod=fortran cppcod=c++ sys=console

\end{Verbatim}

that maps environments (\texttt{xxx}) onto valid language types for
Pygments (which is what \texttt{sphinx} applies to typeset computer code).
The \texttt{xxx} specifier has only effect for \texttt{latex} and
\texttt{sphinx} output. All other formats use a fixed monospace font for all
kinds of verbatim output.
Here is an example of computer code (see the source of this document
for exact syntax):
\begin{Verbatim}[numbers=none,fontsize=\fontsize{9pt}{9pt},baselinestretch=0.95,xleftmargin=2mm]
from numpy import sin, cos, exp, pi

def f(x, y, z, t):
    return exp(-t)*sin(pi*x)*sin(pi*y)*cos(2*pi*z)

\end{Verbatim}

% When showing copy from file in !bc envir, indent a character - otherwise
% ptex2tex is confused and starts copying...
Computer code can also be copied from a file:
\begin{Verbatim}[numbers=none,fontsize=\fontsize{9pt}{9pt},baselinestretch=0.95,xleftmargin=2mm]
 @@@CODE doconce_program.sh
 @@@CODE doconce_program.sh  fromto: doconce clean@^doconce split_rst
 @@@CODE doconce_program.sh  from-to: doconce clean@^doconce split_rst
 @@@CODE doconce_program.sh  envir=shpro fromto: name=@

\end{Verbatim}

The \texttt{@@@CODE} identifier must appear at the very beginning of the line.
The first line copies the complete file \Verb!doconce_program.sh!.
The second line copies from the first line matching the \emph{regular
expression} \texttt{doconce clean} up to, but not including, the line
matching the \emph{regular expression} \Verb!^doconce split_rst!.
The third line behaves as the second, but the line matching
the first regular expression is not copied (this construction is often
used for copying text between begin-end comment pair in the file).
The copied lines from file are in this example put inside \Verb?!bc shpro?
and \Verb?!ec? directives, if a complete file is copied, while the
directives become \Verb?!bc shcod? and \Verb?!ec? when a code snippet is copied
from a file. In general, for a filename extension \texttt{.X}, the environment
becomes \Verb?!bc Xpro? or \Verb?!bc Xcod? for a complete program or snippet,
respectively. The enivorments (\texttt{Xcod} and \texttt{Xpro}) are only active for
\texttt{latex}, \texttt{pdflatex}, html`, and \texttt{sphinx} output.  The fourth line
above specifies the code environment explicitly (\texttt{envir=shpro}) such
that it indicates a complete shell program (\texttt{shpro}) even if we copy a
part of the file (here from \texttt{name=} until the end of the file).
Copying a part of a file will by default lead to \Verb?!bc shcod?, which indicates a
code snippet that normally needs more code to run properly.
The \Verb!--code_prefix=text! option adds a path \texttt{text} to the filename specified
in the \texttt{@@@CODE} command (URLs work). For example
\begin{Verbatim}[numbers=none,fontsize=\fontsize{9pt}{9pt},baselinestretch=0.95,xleftmargin=2mm]
 @@@CODE src/myfile.py

\end{Verbatim}

and \Verb!--code_prefix=http://some.place.net!, the file
\begin{Verbatim}[numbers=none,fontsize=\fontsize{9pt}{9pt},baselinestretch=0.95,xleftmargin=2mm]
http://some.place.net/src/myfile.py

\end{Verbatim}

will be included. If source files have a header with author, email, etc.,
one can remove this header by the option \Verb!'--code_skip_until=# ---!.
The lines up to and including (the first) \Verb!# ---! will then be excluded.
Important warnings:
\begin{itemize}
 \item A code block must come after a plain sentence (at least for successful
   output in reStructredText), not directly after a section/paragraph heading,
   table, comment, figure, or movie.
 \item Verbatim code blocks inside lists can be ugly when typeset in some
   output formats. A more robust approach is to replace the list with
   paragraphs that include headings.
\end{itemize}
\noindent
\subsection{{\LaTeX} Mathematics}
DocOnce supports inline mathematics and blocks of mathematics, using
standard {\LaTeX} syntax. The output formats \texttt{html}, \texttt{sphinx}, \texttt{latex},
pdflatex`, \texttt{pandoc}, and \texttt{mwiki} work with this syntax while all other
formats will just display the raw {\LaTeX} code.
Inline expressions are written in the standard
{\LaTeX} way with the mathematics surrounded by dollar signs, as in
$Ax=b$. To help increase readability in other formats than \texttt{sphinx},
\texttt{latex}, and \texttt{pdflatex}, inline mathematics may have a more human
readable companion expression. The syntax is like
\begin{Verbatim}[numbers=none,fontsize=\fontsize{9pt}{9pt},baselinestretch=0.95,xleftmargin=2mm]
$\sin(\norm{\bf u})$|$sin(||u||)$

\end{Verbatim}

That is, the {\LaTeX} expression appears to the left of a vertical bar (pipe
symbol) and the more readable expression appears to the right. Both
expressions are surrounded by dollar signs.
Blocks of {\LaTeX} mathematics are written within
\Verb?!bt?
and
\Verb?!et? (begin/end TeX) directives starting on the beginning of a line:
\begin{Verbatim}[numbers=none,fontsize=\fontsize{9pt}{9pt},baselinestretch=0.95,xleftmargin=2mm]
!bt
\begin{align*}
\nabla\cdot \pmb{u} &= 0,\\ 
\nabla\times \pmb{u} &= 0.
\end{align*}
!et

\end{Verbatim}

This {\LaTeX} code gets rendered as
\begin{align*}
\nabla\cdot \pmb{u} &= 0,\\ 
\nabla\times \pmb{u} &= 0.
\end{align*}
Here is a single equation:
\begin{Verbatim}[numbers=none,fontsize=\fontsize{9pt}{9pt},baselinestretch=0.95,xleftmargin=2mm]
!bt
\[ \frac{\partial\pmb{u}}{\partial t} + \pmb{u}\cdot\nabla\pmb{u} = 0.\]
!et

\end{Verbatim}

which results in
\[ \frac{\partial\pmb{u}}{\partial t} + \pmb{u}\cdot\nabla\pmb{u} = 0.\]
\paragraph{LaTeX Newcommands.}
The author can define \texttt{newcommand} statements in files with names
\texttt{newcommands*.tex}. Such commands should only be used for mathematics
(other {\LaTeX} constructions are only understood by {\LaTeX} itself).
The convention is that \Verb!newcommands_keep.tex!
contains the newcommands that are kept in the document, while
those in \Verb!newcommands_replace.tex! will be replaced by their full
{\LaTeX} code. This conventions helps make readable documents in formats
without {\LaTeX} support. For \texttt{html}, \texttt{sphinx}, \texttt{latex}, \texttt{pdflatex},
\texttt{mwiki}, \texttt{ipynb}, and \texttt{pandoc}, the mathematics in newcommands is
rendered nicely anyway. If you desire \texttt{newcommand} outside {\LaTeX}
mathematics, simply use a Mako variable or a Mako function (which
will be much more flexible and powerful).
% include with mako must be in this root tree, so we need a link, see make.sh
\subsection{Writing Guidelines (Especially for {\LaTeX} Users!)}
\label{manual:latex:guide}
{\LaTeX} writers often have their own writing habits and have preferred
{\LaTeX} packages. DocOnce is a simpler format and
corresponds to writing in quite plain {\LaTeX} and making the ascii text
look nice (be careful with the use of white space!). This means that
although DocOnce has borrowed a lot from {\LaTeX}, there are a few points
{\LaTeX} writers should pay attention to. Experience shows that these
points are so important that we list them \emph{before} we list typical
DocOnce syntax!
Any {\LaTeX} syntax in mathematical formulas is accepted when DocOnce
translates the text to {\LaTeX}, but the following rules should be
followed when transalting the text to \texttt{sphinx}, \texttt{pandoc},
\texttt{mwiki}, \texttt{html}, or \texttt{ipynb} formats.
\begin{itemize}
 \item AMS {\LaTeX} mathematics is supported, also for the \texttt{html}, \texttt{sphinx},
   and \texttt{ipynb} formats.
 \item If you want {\LaTeX} math blocks to work with \texttt{latex}, \texttt{html}, \texttt{sphinx},
   \texttt{markdown}, and \texttt{ipynb}, only use
   the following equation environments: \Verb!\[ ... \]!,
   \texttt{equation*}, \texttt{equation}, \texttt{align*}, \texttt{align}. \texttt{alignat*}, \texttt{alignat}.
   Other environments, such as \texttt{split}, \texttt{multiline}, \texttt{gather} are
   supported in modern MathJax in HTML and Sphinx, but may have
   rendering problems (to a larger extent than \texttt{equation} and \texttt{align}).
   DocOnce performs extensions to \texttt{sphinx}, \texttt{ipynb},
   and other formats such that
   labels in \texttt{align} and \texttt{alignat} environments work well.
   If you face problems with fancy {\LaTeX} equation environments in
   web formats, try rewriting with plain \texttt{align}, \texttt{nonumber}, etc.
 \item Do not use comments inside equations.
 \item Newcommands in mathematical formulas are allowed, but not in
   the running text. Newcommands must be defined in files with names
   \texttt{newcommands*.tex}. Use \Verb!\newcommands! and not \Verb!\def!.
   Each newcommand must be defined on a single line.
   Use Mako functions if you need macros in the running text.
 \item Use labels and refer to them
   for sections, figures, movies, and equations only.
   MediaWiki (\texttt{mwiki}) does not support references to equations.
 \item Spaces are not allowed in labels.
 \item There is just one \texttt{ref} command (no \Verb!\eqref! for equations) and references to
   equations must use parentheses. Never use the tilde \texttt{~} (non-breaking
   space) character before references to figures, sections, etc., but
   tilde is allowed for references to equations.
 \item Never use \Verb!\pageref! as pages are not a concept in web documents
   (there is only a \texttt{ref} command in DocOnce and it refers to labels).
 \item Only figures and movies are floating elements in DocOnce, all other elements
   (code, tables, algorithms) must appear \emph{inline} without numbers or labels
   for reference\footnote{There is an exception: by using \emph{user-defined environments} within \Verb?!bu-name? and \Verb?!eu-name? directives, it is possible to label any type of text and refer to it. For example, one can have environments for examples, tables, code snippets, theorems, lemmas, etc. One can also use Mako functions to implement environments.} (refer to inline elements by a section label).
   The reason is that floating elements are in general
   not used in web documents, but we made an exception with figures
   and movies.
 \item Keep figure captions shorts as they are used as references in the
   Sphinx format. Avoid inline mathematics since Sphinx will strip it
   away in the figure reference.
   (Many writing styles encourage rich captions that
   explains everything about the figure; this work well
   only in the HTML and {\LaTeX} formats.)
 \item You cannot use \texttt{subfigure} to combine several image files in one
   figure, but you can combine the files into one file using
   the \Verb!doconce combine_images! tool. Refer to individual image files
   in the caption or text by (e.g.) ``left'' and ``right'', or
   ``upper left'', ``lower right'', etc.
 \item Footnotes can be used as usual in {\LaTeX}, but some HTML formats are not
   able to display mathematics or inline verbatim or other formatted
   code (emphasis, boldface, color) in footnotes - keep that in mind.
 \item Use plain \texttt{cite} for references (e.g., \Verb!\citeauthor! has no counterpart
   in DocOnce). The bibliography must be prepared in the Publish format,
   but import from (clean) \textsc{Bib}\negthinspace{\TeX} is possible.
 \item Use \texttt{idx} for index entries, but put the definitions between paragraphs,
   not inside them (required by Sphinx).
 \item Use the \Verb!\bm! command (from the \texttt{bm} package, always included by DocOnce)
   for boldface in mathematics.
 \item Make sure all ordinary text starts in column 1 on each line. Equations
   can be indented. The \Verb!\begin{}! and \Verb!\end{}! directives should start
   in column 1.
 \item If you depend on various {\LaTeX} environments for your writings, you have
   to give these up, or implement \emph{user-defined environments} in DocOnce.
   For instance, examples are normally typeset as subsections in DocOnce,
   but can also utilize a user-defined example environment.
   Learn about the exercise support in DocOnce for typesetting
   exercises, problems, and projects.
 \item Learn about the preprocessors Preprocess and Mako - these are smart
   tools for, e.g., commenting out/in large portions of text and creating
   macros.
 \item Use \emph{generalized references} when referring to companion documents
   that may later become part of this document (or migrated out of this document).
 \item Follow \href{{https://github.com/doconce/setup4book-doconce}}{recommendations for DocOnce books} if you plan to write a book.
\end{itemize}
\noindent

\begin{notice_mdfboxadmon}[Use the preprocessor to tailor output.]
If you really need special {\LaTeX} constructs in the {\LaTeX} output
from DocOnce, you may use use preprocessor if-tests on
the format (typically \Verb!#if FORMAT in ("latex", "pdflatex")!) to
include such special {\LaTeX} code. With an else clause you can easily
create corresponding constructions for other formats. This way
of using Preprocess or Mako
allows you to use advanced {\LaTeX} features (or HTML features for the HTML formats)
to fine tune the resulting
document. More tuning can be done by automatic editing of the
output file (e.g., \texttt{.tex} or \texttt{.html}) produced by DocOnce using
your own scripts or the \texttt{doconce replace} and \texttt{doconce subst} commands.
\end{notice_mdfboxadmon} % title: Use the preprocessor to tailor output.



\begin{notice_mdfboxadmon}[Autotranslation of {\LaTeX} to DocOnce?]
The tool \texttt{doconce latex2doconce} may help you translate {\LaTeX} files
to DocOnce syntax. However, if you use computer code in floating
list environments, special packages for typesetting algorithms,
example environments, \texttt{subfigure} in figures,
or a lot of newcommands in the running text, there will be need for
a lot of manual edits and adjustments.
For examples, figure environments can only be translated by
\texttt{doconce latex2doconce} if the label is inside the caption and
the figure is typeset like
\begin{Verbatim}[numbers=none,fontsize=\fontsize{9pt}{9pt},baselinestretch=0.95,xleftmargin=2mm]
\begin{figure}
  \centering
  \includegraphics[width=0.55\linewidth]{figs/myfig.pdf}
  \caption{This is a figure. \labe{myfig}}
\end{figure}

\end{Verbatim}

If the {\LaTeX} text is consistent with respect to the placement of the label, a
simple script can autoedit the label inside the caption, but many
{\LaTeX} writers put the label at different places in different figures,
and then it becomes more difficult to autoedit figures and translate
them to the DocOnce \texttt{FIGURE:} syntax.
Tables are hard to interpret and translate because headings and
caption can be typeset in many different ways. The type of table
that is recognized looks like
\begin{Verbatim}[numbers=none,fontsize=\fontsize{9pt}{9pt},baselinestretch=0.95,xleftmargin=2mm]
\begin{table}
\caption{Here goes the caption.}
\begin{tabular}{lr}
\hline
\multicolumn{1}{c}{$v_0$} & \multicolumn{1}{c}{$f_R(v_0)$}\\hline
1.2 & 4.2\1.1 & 4.0\0.9 & 3.7
\hline
\end{tabular}
\end{table}

\end{Verbatim}

Recall that table captions do not make sense in DocOnce since tables
must be inlined and explained in the surrounding text.
Footnotes are also problematic for \texttt{doconce latex2doconce} since DocOnce
footnotes must have the explanation outside the paragraph where the
footnote is used. This calls for manual work. The translator from
{\LaTeX} to DocOnce will insert \Verb!_PROBLEM_! and mark footnotes. One
solution is to avoid footnotes in the {\LaTeX} document if fully automatic
translation is desired.
\end{notice_mdfboxadmon} % title: Autotranslation of {\LaTeX} to DocOnce?


\subsection{Hyperlinks}
Links use either a link text or the raw URL:
\begin{Verbatim}[numbers=none,fontsize=\fontsize{9pt}{9pt},baselinestretch=0.95,xleftmargin=2mm]
Here is some "some link text": "http://some.net/address"
(as in "search google": "http://google.com")
or just the raw address: URL: "http://google.com".

Links to files typeset in verbatim mode applies backtics:
"`myfile.py`": "http://some.net/some/place/myfile.py".

Mail addresses works too: send problems to
"`hpl@simula.no`": "mailto:hpl@simula.no"
or just "send mail": "mailto:hpl@simula.no".

\end{Verbatim}

\subsection{Figures and Movies}
Figures and movies have almost equal syntax:
\begin{Verbatim}[numbers=none,fontsize=\fontsize{9pt}{9pt},baselinestretch=0.95,xleftmargin=2mm]
FIGURE: [relative/path/to/figurefile, width=500 frac=0.8] Here goes the caption which must be on a single line. label{some:fig:label}

MOVIE: [relative/path/to/moviefile, width=500] Here goes the caption which must be on a single line. 


\end{Verbatim}

Note three important syntax details:
\begin{enumerate}
 \item A mandatory comma after the figure/movie filename,
 \item no comments between \texttt{width}, \texttt{height}, and \texttt{frac} and no spaces
    around the \texttt{=} characters,
 \item all of the commands must appear on a single line,
 \item there must be a blank line after the command.
\end{enumerate}
\noindent
The figure file can be listed without extension. DocOnce will then find
the version of the file with the most appropriate extension for the chosen
output format. If not suitable version is found, DocOnce will convert
another format to the needed one.
The caption is optional. If omitted, the figure will be inlined (meaning
no use of any figure environment in HTML or {\LaTeX} formats). The \texttt{width}
and \texttt{height} parameters affect HTML formats (\texttt{html}, \texttt{rst}, \texttt{sphinx}),
while \texttt{frac} is the width of the image as a fraction of the total text
width in the \texttt{latex} and \texttt{pdflatex} formats.
The command-line options \Verb!--fig_prefix=...! and \Verb!--mov_prefix=...! can
be used to add a path (can be a URL) to all figure and movie files,
respectively.  This is useful when including DocOnce documents in
other DocOnce documents such that the text is compiled in different
directories (with different paths to the figure directory).
Movie files can either be a video or a wildcard expression for a
series of frames. In the latter case, a simple device in an HTML page
will display the individual frame files as a movie.
Combining several image files into one can be done by the
\begin{Verbatim}[numbers=none,fontsize=\fontsize{9pt}{9pt},baselinestretch=0.95,xleftmargin=2mm]
doconce combine_images image1 image2 ... output_image

\end{Verbatim}

This command applies \texttt{montage} or PDF-based tools to combine the images
to get the highest quality.
YouTube and Vimeo movies will be embedded in \texttt{html} and \texttt{sphinx} documents
and otherwise be represented by a link. The syntax is
\begin{Verbatim}[numbers=none,fontsize=\fontsize{9pt}{9pt},baselinestretch=0.95,xleftmargin=2mm]
MOVIE: [http://www.youtube.com/watch?v=_O7iUiftbKU, width=420 height=315] YouTube movie.

MOVIE: [http://vimeo.com/55562330, width=500 height=278] Vimeo movie.


\end{Verbatim}

The latter results in
\begin{doconce:movie}
\refstepcounter{doconce:movie:counter}
\begin{center}\href{{http://vimeo.com/55562330}}{\nolinkurl{http://vimeo.com/55562330}}\end{center}
\begin{center}  % movie caption
Movie \arabic{doconce:movie:counter}: Vimeo movie.
\end{center}
\end{doconce:movie}
\subsection{Tables}
The table in Section~\ref{quick:sections} was written with this
syntax:
\begin{Verbatim}[numbers=none,fontsize=\fontsize{9pt}{9pt},baselinestretch=0.95,xleftmargin=2mm]

|----------------c--------|------------------c--------------------|
|      Section type       |        Syntax                         |
|----------------l--------|------------------l--------------------|
| chapter                 | `========= Heading ========` (9 `=`)  |
| section                 | `======= Heading =======`    (7 `=`)  |
| subsection              | `===== Heading =====`        (5 `=`)  |
| subsubsection           | `=== Heading ===`            (3 `=`)  |
| paragraph               | `__Heading.__`               (2 `_`)  |
|-----------------------------------------------------------------|


\end{Verbatim}

Note that
\begin{itemize}
 \item Each line begins and ends with a vertical bar (pipe symbol).
 \item Column data are separated by a vertical bar (pipe symbol).
 \item There must be a blank line before and after the table.
 \item There may be horizontal rules, i.e., lines with dashes for
   indicating the heading and the end of the table, and these may
   contain characters 'c', 'l', or 'r' for how to align headings or
   columns. The first horizontal rule may indicate how to align
   headings (center, left, right), and the horizontal rule after the
   heading line may indicate how to align the data in the columns
   (center, left, right).
   One can also use \texttt{X} for potentially very wide text that must be
   wrapped and left-adjusted (will only affect \texttt{latex} and \texttt{pdflatex} where the
   \texttt{tabularx} package is then used; \texttt{X} means \texttt{l} in all other formats).
 \item If the horizontal rules are without alignment information there should
   be no vertical bar (pipe symbol) between the columns. Otherwise, such
   a bar indicates a vertical bar between columns in {\LaTeX}.
 \item Many output formats are so primitive that heading and column alignment
   have no effect.
\end{itemize}
\noindent
A quick way of generating tables is to place all the entries in a file
with comma as separator (a CSV file) and then run the utility
\texttt{doconce csv2table} to create a table in the DocOnce format.
The command-line option \texttt{--tables2csv} (to \texttt{doconce format})
makes DocOnce dump each table to CSV format in a file \Verb!table_X.csv!,
where \texttt{X} is the table number. This feature makes it easy to
load tables into spreadsheet programs for further analysis.
DocOnce tables can be efficiently made directly from data in CSV files.
\begin{Verbatim}[numbers=none,fontsize=\fontsize{9pt}{9pt},baselinestretch=0.95,xleftmargin=2mm]
Terminal> doconce csv2table mydata.csv > mydata_table.do.txt

\end{Verbatim}

Now we can do \Verb!# #include "mydata_table.do.txt"! in the DocOnce
source file or simply copy the table in \Verb!mydata_table.do.txt!
into the DocOnce file.
\subsection{Labels and References}
The notion of labels and references (as well as bibliography and index)
is adopted
from {\LaTeX} with a very similar syntax. As in {\LaTeX}, a label can be
inserted anywhere, using the syntax
\begin{Verbatim}[numbers=none,fontsize=\fontsize{9pt}{9pt},baselinestretch=0.95,xleftmargin=2mm]
label{name}

\end{Verbatim}

with no backslash
preceding the label keyword. It is common practice to choose \texttt{name}
as some hierarchical name, using a delimiter like \texttt{:} or \Verb!_! between
(e.g.) section, subsection, and topic.
A reference to the label \texttt{name} is written as
\begin{Verbatim}[numbers=none,fontsize=\fontsize{9pt}{9pt},baselinestretch=0.95,xleftmargin=2mm]
ref{name}

\end{Verbatim}

again with no backslash before \texttt{ref}.
Use labels for sections and equations only, and precede the reference
by "Section" or "Chapter", or in case of an equation, surround the
reference by parenthesis.
\subsection{Citations and Bibliography}
Single citations are written as
\begin{Verbatim}[numbers=none,fontsize=\fontsize{9pt}{9pt},baselinestretch=0.95,xleftmargin=2mm]
cite{name}

\end{Verbatim}

where \texttt{name} is a logical name
of the reference (again, {\LaTeX} writers must not insert a backslash).
Bibliography citations often have \texttt{name} on the form
\Verb!Author1_Author2_YYYY!, \Verb!Author_YYYY!, or \Verb!Author1_etal_YYYY!, where
\texttt{YYYY} is the year of the publication.
Multiple citations at once is possible by separating the logical names
by comma:
\begin{Verbatim}[numbers=none,fontsize=\fontsize{9pt}{9pt},baselinestretch=0.95,xleftmargin=2mm]
cite{name1,name2,name3}

\end{Verbatim}

The bibliography is specified by a line \texttt{BIBFILE: papers.pub},
where \texttt{papers.pub} is a publication database in the
\href{{https://github.com/doconce/publish}}{Publish} format.
\textsc{Bib}\negthinspace{\TeX} \texttt{.bib} files can easily be combined to a Publish database
(which DocOnce needs to create bibliographies in other formats
than {\LaTeX}).
\subsection{Generalized References}
There is a \emph{generalized referencing} feature in DocOnce that allows
a reference with \texttt{ref} to have one formulation if the label is
in the same document and another formulation if the reference is
to an item in an external document. This construction makes it easy
to work with many small, independent documents in parallel with
a book assembly of some of the small elements.
The syntax of a generalized reference is
\begin{Verbatim}[numbers=none,fontsize=\fontsize{9pt}{9pt},baselinestretch=0.95,xleftmargin=2mm]
ref[internal][cite][external]

\end{Verbatim}

with a specific example being
\begin{Verbatim}[numbers=none,fontsize=\fontsize{9pt}{9pt},baselinestretch=0.95,xleftmargin=2mm]
As explained in
ref[Section ref{subsec:ex}][in cite{testdoc:12}][a "section":
"testdoc.html#___sec2" in the document
"A Document for Testing DocOnce": "testdoc.html" cite{testdoc:12}],
DocOnce documents may include movies.

\end{Verbatim}

The output from a generalized reference is the text \texttt{internal} if all
references with \texttt{ref} in the text \texttt{internal} are references to labels
defined in the
present document. Otherwise, if \texttt{cite} is non-empty and the format is
\texttt{latex} or \texttt{pdflatex}, one assumes that the references in \texttt{internal}
are to external documents declared by a comment line
\Verb!# Externaldocuments: testdoc, mydoc! (usually after the title, authors,
and date). In this case the output text is \texttt{internal} followed by \texttt{cite},
and the
{\LaTeX} package \texttt{xr} is used to handle the labels in the external
documents.  If none of the two situations above applies, the
\texttt{external} text will be the output.
\subsection{Index of Keywords}
DocOnce supports creating an index of keywords. A certain keyword
is registered for the index by a syntax like (no
backslash!)
\begin{Verbatim}[numbers=none,fontsize=\fontsize{9pt}{9pt},baselinestretch=0.95,xleftmargin=2mm]
index{name}

\end{Verbatim}

It is recommended to place any index of this type outside
running text, i.e., after (sub)section titles and in the space between
paragraphs. Index specifications placed right before paragraphs also
gives the doconce source code an indication of the content in the
forthcoming text. The index is only produced for the \texttt{latex},
\texttt{pdflatex}, \texttt{rst}, and \texttt{sphinx} formats.
\subsection{Capabilities of The Program \texttt{doconce} }
The \texttt{doconce} program can be used for a number of purposes besides
transforming a \texttt{.do.txt} file to some format. Here is the
list of capabilities:
\begin{Verbatim}[numbers=none,fontsize=\fontsize{9pt}{9pt},baselinestretch=0.95,xleftmargin=2mm]
DocOnce version X.X.X
Usage: doconce command [optional arguments]
commands: help format find subst replace remove spellcheck apply_inline_edits capitalize change_encoding clean combine_images csv2table diff expand_commands expand_mako extract_exercises find_nonascii_chars fix_bibtex4publish gitdiff grab grep guess_encoding gwiki_figsubst html2doconce html_colorbullets jupyterbook include_map insertdocstr ipynb2doconce latex2doconce latex_dislikes latex_exercise_toc latex_footer latex_header latex_problems latin2html lightclean linkchecker list_fig_src_files list_labels makefile md2html md2latex old2new_format ptex2tex pygmentize ref_external remove_exercise_answers remove_inline_comments replace_from_file slides_beamer slides_html slides_markdown sphinx_dir sphinxfix_localURLs split_html split_rst teamod

doconce format html|latex|pdflatex|rst|sphinx|plain|gwiki|mwiki|
               cwiki|pandoc|st|epytext dofile 
# transform doconce file to another format

doconce subst [-s -m -x --restore] regex-pattern \ 
        regex-replacement file1 file2 ... 
# substitute a phrase by another using regular expressions (in this example -s is the re.DOTALL modifier, -m is the re.MULTILINE modifier, -x is the re.VERBOSE modifier, --restore copies backup files back again)

doconce replace from-text to-text file1 file2 ...                      
# replace a phrase by another literally (exact text substitution)

doconce replace_from_file file-with-from-to-replacements file1 file2 ... 
# replace using from and to phrases from file

doconce find expression                                                
# search for a (regular) expression in all .do.txt files in the current directory tree (useful when removing compilation errors)

doconce include_map mydoc.do.txt                                       
# print an overview of how various files are included in the root doc

doconce expand_mako mako_code_file funcname file1 file2 ...            
# replace all mako function calls by the `results of the calls

doconce remove_inline_comments dofile                                  
# remove all inline comments in a doconce file

doconce apply_inline_edits                                             
# apply all edits specified through inline comments

doconce sphinx_dir copyright='John Doe' title='Long title' \
        short_title="Short title" version=0.1 intersphinx \
        /path/to/mylogo.png dofile 
# create a directory for the sphinx format (requires sphinx version >= 1.1)

doconce format sphinx complete_file 
doconce split_rst complete_file 
doconce sphinx_dir complete_file 
python automake_sphinx.py 
# split a sphinx/rst file into parts according to !split commands

doconce insertdocstr rootdir                                           
# walk through a directory tree and insert doconce files as docstrings in *.p.py files

doconce lightclean                                                     
# remove all redundant files (keep source .do.txt and results: .pdf, .html, sphinx- dirs, .mwiki, .ipynb, etc.)

doconce clean                                                          
# remove all files that the doconce can regenerate

doconce change_encoding utf-8 latin1 dofile                            
# change encoding

doconce guess_encoding filename                                        
# guess the encoding in a text

doconce find_nonascii_chars file1 file2 ...                            
# find non-ascii characters in a file

doconce split_html complete_file.html                                  
# split an html file into parts according to !split commands

doconce slides_html slide_type complete_file.html                      
# create HTML slides from a (doconce) html file

doconce slides_beamer complete_file.tex                                
# create LaTeX Beamer slides from a (doconce) latex/pdflatex file

doconce slides_markdown complete_file.md remark --slide_style=light    
# create Remark slides from Markdown

doconce html_colorbullets file1.html file2.html ...                    
# replace bullets in lists by colored bullets

doconce extract_exercises tmp_mako__mydoc                              
# extract all exercises (projects and problems too)

doconce grab --from[-] from-text [--to[-] to-text] file > result       
# grab selected text from a file

doconce remove --from[-] from-text [--to[-] to-text] file > result     
# remove selected text from a file

doconce grep FIGURE|MOVIE|CODE dofile                                  
# list all figure, movie or included code files

doconce spellcheck [-d .mydict.txt] *.do.txt                           
# run spellcheck on a set of files

doconce ptex2tex mydoc -DMINTED pycod=minted sys=Verbatim \
        dat=\begin{quote}\begin{verbatim};\end{verbatim}\end{quote} 
# transform ptex2tex files (.p.tex) to ordinary latex file and manage the code environments

doconce md2html file.md                                                
# make HTML file via pandoc from Markdown (.md) file

doconce md2latex file.md                                               
# make LaTeX file via pandoc from Markdown (.md) file

doconce combine_images image1 image2 ... output_file                   
# combine several images into one

doconce latex_problems mydoc.log [overfull-hbox-limit]                 
# report problems from a LaTeX .log file

doconce list_fig_src_files *.do.txt                                    
# list all figure files, movie files, and source code files needed

doconce list_labels myfile                                             
# list all labels in a document (for purposes of cleaning them up)

doconce ref_external mydoc [pubfile]                                   
# generate script for substituting generalized references

doconce linkchecker *.html                                             
# check all links in HTML files

doconce capitalize [-d .mydict.txt] *.do.txt                           
# change headings from "This is a Heading" to "This is a heading"

doconce latex2doconce latexfile                                        
# translate a latex document to doconce (requires usually manual fixing)

doconce latex_dislikes latexfile                                       
# check if there are problems with translating latex to doconce

doconce ipynb2doconce notebookfile                                     
# translate an IPython/Jupyter notebook to doconce

doconce pygmentize myfile [pygments-style]                             
# typeset a doconce document with pygments (for pretty print of doconce itself)

doconce makefile docname doconcefile [html sphinx pdflatex ...]        
# generate a make.py script for translating a doconce file to various formats

doconce diff file1.do.txt file2.do.txt [diffprog]                      
# find differences between two files (diffprog can be difflib, diff, pdiff, latexdiff, kdiff3, diffuse, ...)

doconce gitdiff file1 file2 file3 ...                                  
# find differences between the last two Git versions of several files

doconce csv2table somefile.csv                                         
# convert csv file to doconce table format

doconce sphinxfix_local_URLs file.rst                                  
# edit URLs to local files and place them in _static

doconce latin2html file.html                                           
# replace latex-1 (non-ascii) characters by html codes

doconce fix_bibtex4publish file1.bib file2.bib ...                     
# fix common problems in bibtex files for publish import

doconce latex_header                                                   
# print the header (preamble) for latex file

doconce latex_footer                                                   
# print the footer for latex files

doconce expand_commands file1 file2 ...                                
# expand short cut commands to full form in files

doconce latex_exercise_toc myfile                                      
# insert a table of exercises in a latex file myfile.p.tex

\end{Verbatim}

\subsection{Exercises}
DocOnce supports \emph{Exercise}, \emph{Problem}, \emph{Project}, and \emph{Example}.
These are typeset
as ordinary sections and referred to by their section labels.
Exercise, problem, project, or example sections contains certain \emph{elements}:
\begin{itemize}
  \item a headline at the level of a subsection
    containing one of the words "Exercise:", "Problem:",
    "Project:", or "Example:", followed by a title (required)
  \item a label (optional)
  \item a solution file (optional)
  \item name of file with a student solution (optional)
  \item main exercise text (required)
  \item a short answer (optional)
  \item a full solution (optional)
  \item one or more hints (optional)
  \item one or more subexercises (subproblems, subprojects), which can also
    contain a text, a short answer, a full solution, name student file
    to be handed in, and one or more hints (optional)
\end{itemize}
\noindent
A typical sketch of a a problem without subexercises goes as follows:
\begin{Verbatim}[numbers=none,fontsize=\fontsize{9pt}{9pt},baselinestretch=0.95,xleftmargin=2mm]
===== Problem: Derive the Formula for the Area of an Ellipse =====
label{problem:ellipsearea1}
file=ellipse_area.pdf
solution=ellipse_area1_sol.pdf

Derive an expression for the area of an ellipse by integrating
the area under a curve that defines half of the ellipse.
Show each step in the mathematical derivation.

!bhint
Wikipedia has the formula for the curve.
!ehint

!bhint
"Wolframalpha": "http://wolframalpha.com" can perhaps
compute the integral.
!ehint

\end{Verbatim}

If the exercise type (Exercise, Problem, Project, or Example)
is enclosed in braces, the type is left out of the title in the
output. For example, the if the title line above reads
\begin{Verbatim}[numbers=none,fontsize=\fontsize{9pt}{9pt},baselinestretch=0.95,xleftmargin=2mm]
===== {Problem}: Derive the Formula for the Area of an Ellipse =====

\end{Verbatim}

the title becomes just "Derive the ...".
An exercise with subproblems, answers and full solutions has this
setup-up:
\begin{Verbatim}[numbers=none,fontsize=\fontsize{9pt}{9pt},baselinestretch=0.95,xleftmargin=2mm]
===== Exercise: Determine the Distance to the Moon =====
label{exer:moondist}

Intro to this exercise. Questions are in subexercises below.

!bsubex
Subexercises are numbered a), b), etc.

file=subexer_a.pdf

!bans
Short answer to subexercise a).
!eans

!bhint
First hint to subexercise a).
!ehint

!bhint
Second hint to subexercise a).
!ehint
!esubex

!bsubex
Here goes the text for subexercise b).

file=subexer_b.pdf

!bhint
A hint for this subexercise.
!ehint

!bsol
Here goes the solution of this subexercise.
!esol
!esubex

!bremarks
At the very end of the exercise it may be appropriate to summarize
and give some perspectives. The text inside the `!bremarks` and `!eremarks`
directives is always typeset at the end of the exercise.
!eremarks

!bsol
Here goes a full solution of the whole exercise.
!esol


\end{Verbatim}

By default, answers, solutions, and hints are typeset as paragraphs.
The command-line arguments \Verb!--without_answers! and \Verb!--without_solutions!
turn off output of answers and solutions, respectively, except for examples. The command line options \Verb!--answers_at_end! and \Verb!--solutions_at_end! write all answers and solutions to exercises to a separate section at the end of the document, respectively. Combine with \Verb!--without_answers! and \Verb!--without_solutions! to remove answers and solutions from the main text.
The commands \Verb?!anshide? and \Verb?!solhide? can be used to hide from the main text answers and solutions, respectively, until the \Verb?!ansoff? and \Verb?!soloff? commands are encountered. Similarly, the \Verb?!ansdocend? and \Verb?!soldocend? commands move answers and solutions to the end of the book. 
\subsection{Environments}
DocOnce environments start with \Verb?!benvirname? and end with \Verb?!eenvirname?,
where \texttt{envirname} is the name of the environment. Here is a listing of
the environments:
\begin{itemize}
 \item \texttt{c}: computer code (or verbatim text)
 \item \texttt{t}: math blocks with {\LaTeX} syntax
 \item \texttt{subex}: sub-exercise
 \item \texttt{ans}: short answer to exercise or sub-exercise
 \item \texttt{sol}: full solution to exercise or sub-exercise
 \item \texttt{hint}: multiple help items in an exercise or sub-exercise
 \item \texttt{quote}: indented text
 \item \texttt{notice}, \texttt{summary}, \texttt{warning}, \texttt{question}: admonition boxes with
    custom title, special icon, and (sometimes) background color
 \item \texttt{block}, \texttt{box}: simpler boxes (\texttt{block} may have title but never any icon)
 \item \texttt{pop}: text to gradually pop up in slide presentations
 \item \texttt{slidecell}: indication of cells in a grid layout for elements on a
   slide
\end{itemize}
\noindent
In addition, the user can define new environments \Verb?!bc-name? as
explained in the \href{{http://doconce.github.io/doconce/doc/pub/manual/manual.html#manual:userdef:envir}}{manual}.
\subsection{Preprocessing}
DocOnce documents may utilize a preprocessor, either \texttt{preprocess} and/or
\texttt{mako}. The former is a C-style preprocessor that allows if-tests
and including other files (but not macros with arguments).
The \texttt{mako} preprocessor is much more advanced - it is actually a full
programming language, very similar to Python.
The command \texttt{doconce format} first runs \texttt{preprocess} and then \texttt{mako}.
Here is a typical example on utilizing \texttt{preprocess} to include another
document, ``comment out'' a large portion of text, and to write format-specific
constructions:
\begin{Verbatim}[numbers=none,fontsize=\fontsize{9pt}{9pt},baselinestretch=0.95,xleftmargin=2mm]
# #include "myotherdoc.do.txt"

# #if FORMAT in ("latex", "pdflatex")
\begin{table}
\caption{Some words... label{mytab}}
\begin{tabular}{lrr}
\hline\noalign{\smallskip}
\multicolumn{1}{c}{time} & \multicolumn{1}{c}{velocity} & \multicolumn{1}{c}{acceleration} \\ 
\hline
0.0          & 1.4186       & -5.01        \\ 
2.0          & 1.376512     & 11.919       \\ 
4.0          & 1.1E+1       & 14.717624    \\ 
\hline
\end{tabular}
\end{table}
# #else
  |--------------------------------|
  |time  | velocity | acceleration |
  |--l--------r-----------r--------|
  | 0.0  | 1.4186   | -5.01        |
  | 2.0  | 1.376512 | 11.919       |
  | 4.0  | 1.1E+1   | 14.717624    |
  |--------------------------------|
# #endif

# #ifdef EXTRA_MATERIAL
....large portions of text...
# #endif

\end{Verbatim}

With the \texttt{mako} preprocessor the if-else tests have slightly different syntax.
An \href{{http://doconce.github.com/bioinf-py/}}{example document} contains
some illustrations on how to utilize \texttt{mako} (clone the GitHub project and
examine the DocOnce source and the \texttt{doc/src/make.sh} script).
\subsection{Resources}
\begin{itemize}
 \item Excellent "Sphinx Tutorial" by C. Reller: "http://people.ee.ethz.ch/~creller/web/tricks/reST.html"
\end{itemize}
\noindent
% ------------------- end of main content ---------------
% #ifdef PREAMBLE
\end{document}
% #endif

+ name=quickref
+ set -x
+ sh ./clean.sh
Removing in /X/X/quickref:
+ '[' '!' -L guidelines.do.txt ']'
+ doconce
+ sed -i 's/^\x1b\[[0-9;]*m//' doconce_program.sh
+ sed -i 's/\x1b\[[0-9;]*m//' doconce_program.sh
+ sed -i 's/\x1b\[[0-9;]*m\s*/\n# /' doconce_program.sh
+ sed -i 's/\x1b\[[0-9;]*m//g' doconce_program.sh
+ system doconce format html quickref --pygments_html_style=none --no_preprocess --no_abort --html_style=bootswatch_readable
+ doconce format html quickref --pygments_html_style=none --no_preprocess --no_abort --html_style=bootswatch_readable
running mako on quickref.do.txt to make tmp_mako__quickref.do.txt
Translating doconce text in tmp_mako__quickref.do.txt to html
copy complete file doconce_program.sh  (format: shpro)
*** warning: found emphasis tag *...* in footnote, which was removed
    in tooltip (since it does not work with bootstrap tooltips)
    but not in the footnote itself.
There is an exception: by using *user-defined environments* within `!bu-name` and `!eu-name` directives, it is possible to label any type of text and refer to it. For example, one can have environments for examples, tables, code snippets, theorems, lemmas, etc. One can also use Mako functions to implement environments.

*** warning: found inline code tag `...` in footnote, which was removed
    in tooltip (since it does not work with bootstrap tooltips):
There is an exception: by using user-defined environments within `!bu-name` and `!eu-name` directives, it is possible to label any type of text and refer to it. For example, one can have environments for examples, tables, code snippets, theorems, lemmas, etc. One can also use Mako functions to implement environments.

output in quickref.html
+ '[' 0 -ne 0 ']'
+ system doconce format pdflatex quickref --no_preprocess --latex_font=helvetica --no_ampersand_quote --latex_code_style=vrb --no_abort
+ doconce format pdflatex quickref --no_preprocess --latex_font=helvetica --no_ampersand_quote --latex_code_style=vrb --no_abort
running mako on quickref.do.txt to make tmp_mako__quickref.do.txt
Translating doconce text in tmp_mako__quickref.do.txt to pdflatex
copy complete file doconce_program.sh  (format: shpro)
Final all \\label\{(.+?)\}
output in quickref.tex
+ '[' 0 -ne 0 ']'
+ doconce subst '([^`])Guns & Roses([^`])' '\g<1>Guns {\&} Roses\g<2>' quickref.tex
([^`])Guns & Roses([^`]) replaced by \g<1>Guns {\&} Roses\g<2> in quickref.tex
+ doconce subst '([^`])Texas A & M([^`])' '\g<2>Texas A {\&} M\g<2>' quickref.tex
([^`])Texas A & M([^`]) replaced by \g<2>Texas A {\&} M\g<2> in quickref.tex
+ system pdflatex -shell-escape quickref
+ pdflatex -shell-escape quickref
This is pdfTeX, Version 3.14159265-2.6-1.40.18 (TeX Live 2017/Debian) (preloaded format=pdflatex)
 \write18 enabled.
entering extended mode
(./quickref.tex
LaTeX2e <2017-04-15>
Babel <3.18> and hyphenation patterns for 84 language(s) loaded.
(/usr/share/texlive/texmf-dist/tex/latex/base/article.cls
Document Class: article 2014/09/29 v1.4h Standard LaTeX document class



(/usr/share/texlive/texmf-dist/tex/latex/graphics/color.sty



(/usr/share/texlive/texmf-dist/tex/latex/amsmath/amsmath.sty
For additional information on amsmath, use the `?' option.
(/usr/share/texlive/texmf-dist/tex/latex/amsmath/amstext.sty





(/usr/share/texlive/texmf-dist/tex/latex/xcolor/xcolor.sty

(/usr/share/texlive/texmf-dist/tex/latex/colortbl/colortbl.sty


(/usr/share/texlive/texmf-dist/tex/latex/ltablex/ltablex.sty


(/usr/share/texlive/texmf-dist/tex/latex/microtype/microtype.sty



(/usr/share/texlive/texmf-dist/tex/latex/graphics/graphicx.sty
(/usr/share/texlive/texmf-dist/tex/latex/graphics/graphics.sty



(/usr/share/texlive/texmf-dist/tex/latex/fancyvrb/fancyvrb.sty
Style option: `fancyvrb' v2.7a, with DG/SPQR fixes, and firstline=lastline fix 
<2008/02/07> (tvz)) 
 (/usr/share/texlive/texmf-dist/tex/latex/moreverb/moreverb.sty

(/usr/share/texlive/texmf-dist/tex/latex/base/fontenc.sty

(/usr/share/texlive/texmf-dist/tex/latex/ucs/ucs.sty

(/usr/share/texlive/texmf-dist/tex/latex/base/inputenc.sty



(/usr/share/texlive/texmf-dist/tex/latex/hyperref/hyperref.sty
(/usr/share/texlive/texmf-dist/tex/generic/oberdiek/hobsub-hyperref.sty







(/usr/share/texlive/texmf-dist/tex/latex/hyperref/hpdftex.def


(/X/X/mdframed.sty
(/usr/share/texlive/texmf-dist/tex/latex/l3packages/xparse/xparse.sty
(/usr/share/texlive/texmf-dist/tex/latex/l3kernel/expl3.sty



(/usr/share/texlive/texmf-dist/tex/latex/oberdiek/zref-abspage.sty


(/usr/share/texlive/texmf-dist/tex/latex/pgf/frontendlayer/tikz.sty
(/usr/share/texlive/texmf-dist/tex/latex/pgf/basiclayer/pgf.sty
(/usr/share/texlive/texmf-dist/tex/latex/pgf/utilities/pgfrcs.sty
(/usr/share/texlive/texmf-dist/tex/generic/pgf/utilities/pgfutil-common.tex
(/usr/share/texlive/texmf-dist/tex/generic/pgf/utilities/pgfutil-common-lists.t
ex)) (/usr/share/texlive/texmf-dist/tex/generic/pgf/utilities/pgfutil-latex.def


(/usr/share/texlive/texmf-dist/tex/latex/pgf/basiclayer/pgfcore.sty
(/usr/share/texlive/texmf-dist/tex/latex/pgf/systemlayer/pgfsys.sty
(/usr/share/texlive/texmf-dist/tex/generic/pgf/systemlayer/pgfsys.code.tex
(/usr/share/texlive/texmf-dist/tex/generic/pgf/utilities/pgfkeys.code.tex
(/usr/share/texlive/texmf-dist/tex/generic/pgf/utilities/pgfkeysfiltered.code.t
ex)) 
(/usr/share/texlive/texmf-dist/tex/generic/pgf/systemlayer/pgfsys-pdftex.def
(/usr/share/texlive/texmf-dist/tex/generic/pgf/systemlayer/pgfsys-common-pdf.de
f)))
(/usr/share/texlive/texmf-dist/tex/generic/pgf/systemlayer/pgfsyssoftpath.code.
tex)
(/usr/share/texlive/texmf-dist/tex/generic/pgf/systemlayer/pgfsysprotocol.code.
tex))
(/usr/share/texlive/texmf-dist/tex/generic/pgf/basiclayer/pgfcore.code.tex
(/usr/share/texlive/texmf-dist/tex/generic/pgf/math/pgfmath.code.tex
(/usr/share/texlive/texmf-dist/tex/generic/pgf/math/pgfmathcalc.code.tex


(/usr/share/texlive/texmf-dist/tex/generic/pgf/math/pgfmathfunctions.code.tex
(/usr/share/texlive/texmf-dist/tex/generic/pgf/math/pgfmathfunctions.basic.code
.tex)
(/usr/share/texlive/texmf-dist/tex/generic/pgf/math/pgfmathfunctions.trigonomet
ric.code.tex)
(/usr/share/texlive/texmf-dist/tex/generic/pgf/math/pgfmathfunctions.random.cod
e.tex)
(/usr/share/texlive/texmf-dist/tex/generic/pgf/math/pgfmathfunctions.comparison
.code.tex)
(/usr/share/texlive/texmf-dist/tex/generic/pgf/math/pgfmathfunctions.base.code.
tex)
(/usr/share/texlive/texmf-dist/tex/generic/pgf/math/pgfmathfunctions.round.code
.tex)
(/usr/share/texlive/texmf-dist/tex/generic/pgf/math/pgfmathfunctions.misc.code.
tex)
(/usr/share/texlive/texmf-dist/tex/generic/pgf/math/pgfmathfunctions.integerari
thmetics.code.tex)))

(/usr/share/texlive/texmf-dist/tex/generic/pgf/basiclayer/pgfcorepoints.code.te
x)
(/usr/share/texlive/texmf-dist/tex/generic/pgf/basiclayer/pgfcorepathconstruct.
code.tex)
(/usr/share/texlive/texmf-dist/tex/generic/pgf/basiclayer/pgfcorepathusage.code
.tex)
(/usr/share/texlive/texmf-dist/tex/generic/pgf/basiclayer/pgfcorescopes.code.te
x)
(/usr/share/texlive/texmf-dist/tex/generic/pgf/basiclayer/pgfcoregraphicstate.c
ode.tex)
(/usr/share/texlive/texmf-dist/tex/generic/pgf/basiclayer/pgfcoretransformation
s.code.tex)
(/usr/share/texlive/texmf-dist/tex/generic/pgf/basiclayer/pgfcorequick.code.tex
)
(/usr/share/texlive/texmf-dist/tex/generic/pgf/basiclayer/pgfcoreobjects.code.t
ex)
(/usr/share/texlive/texmf-dist/tex/generic/pgf/basiclayer/pgfcorepathprocessing
.code.tex)
(/usr/share/texlive/texmf-dist/tex/generic/pgf/basiclayer/pgfcorearrows.code.te
x)
(/usr/share/texlive/texmf-dist/tex/generic/pgf/basiclayer/pgfcoreshade.code.tex
)
(/usr/share/texlive/texmf-dist/tex/generic/pgf/basiclayer/pgfcoreimage.code.tex

(/usr/share/texlive/texmf-dist/tex/generic/pgf/basiclayer/pgfcoreexternal.code.
tex))
(/usr/share/texlive/texmf-dist/tex/generic/pgf/basiclayer/pgfcorelayers.code.te
x)
(/usr/share/texlive/texmf-dist/tex/generic/pgf/basiclayer/pgfcoretransparency.c
ode.tex)
(/usr/share/texlive/texmf-dist/tex/generic/pgf/basiclayer/pgfcorepatterns.code.
tex)))
(/usr/share/texlive/texmf-dist/tex/generic/pgf/modules/pgfmoduleshapes.code.tex
) (/usr/share/texlive/texmf-dist/tex/generic/pgf/modules/pgfmoduleplot.code.tex
)
(/usr/share/texlive/texmf-dist/tex/latex/pgf/compatibility/pgfcomp-version-0-65
.sty)
(/usr/share/texlive/texmf-dist/tex/latex/pgf/compatibility/pgfcomp-version-1-18
.sty)) (/usr/share/texlive/texmf-dist/tex/latex/pgf/utilities/pgffor.sty
(/usr/share/texlive/texmf-dist/tex/latex/pgf/utilities/pgfkeys.sty

(/usr/share/texlive/texmf-dist/tex/latex/pgf/math/pgfmath.sty

(/usr/share/texlive/texmf-dist/tex/generic/pgf/utilities/pgffor.code.tex

(/usr/share/texlive/texmf-dist/tex/generic/pgf/frontendlayer/tikz/tikz.code.tex

(/usr/share/texlive/texmf-dist/tex/generic/pgf/libraries/pgflibraryplothandlers
.code.tex)
(/usr/share/texlive/texmf-dist/tex/generic/pgf/modules/pgfmodulematrix.code.tex
)
(/usr/share/texlive/texmf-dist/tex/generic/pgf/frontendlayer/tikz/libraries/tik
zlibrarytopaths.code.tex)))
(/X/X/md-frame-1.mdf))

Writing index file quickref.idx
(/usr/share/texlive/texmf-dist/tex/latex/idxlayout/idxlayout.sty

(/usr/share/texlive/texmf-dist/tex/latex/tocbibind/tocbibind.sty

Package tocbibind Note: Using section or other style headings.

) (/usr/share/texlive/texmf-dist/tex/latex/ms/ragged2e.sty

No file quickref.aux.

(/usr/share/texlive/texmf-dist/tex/context/base/mkii/supp-pdf.mkii
[Loading MPS to PDF converter (version 2006.09.02).]
) (/usr/share/texlive/texmf-dist/tex/latex/oberdiek/epstopdf-base.sty



(/usr/share/texlive/texmf-dist/tex/latex/hyperref/nameref.sty

ABD: EveryShipout initializing macros ABD: EverySelectfont initializing macros










Package hyperref Warning: old toc file detected, not used; run LaTeX again.


 [1{/var/lib/texmf/fo
nts/map/pdftex/updmap/pdftex.map}] [2] [3]
Overfull \hbox \(XXXpt too wide\) 
\T1/phv/m/n/10 Note that ab-stracts are rec-og-nized by start-ing with [] or []

[4] [5 <./latex_figs/dizzy_face.png>] [6] [7]
Overfull \hbox \(XXXpt too wide\) 
[]\T1/phv/m/n/10 resulting in Some sen-tence with \T1/lmtt/m/n/10 words in verb
atim style\T1/phv/m/n/10 . Multi-line blocks

Overfull \hbox \(XXXpt too wide\) 
[]\T1/phv/m/n/10 that maps en-vi-ron-ments (\T1/lmtt/m/n/10 xxx\T1/phv/m/n/10 )
 onto valid lan-guage types for Pyg-ments (which

Overfull \hbox \(XXXpt too wide\) 
\T1/phv/m/n/10 ning text. New-com-mands must be de-fined in files with names \T
1/lmtt/m/n/10 newcommands*.tex\T1/phv/m/n/10 .

Overfull \hbox \(XXXpt too wide\) 
\T1/phv/m/n/10 but you can com-bine the files into one file us-ing the []

Overfull \hbox \(XXXpt too wide\) 
\T1/phv/m/n/10 tools for, e.g., com-ment-ing out/in large por-tions of text and
 cre-at-ing macros. 

Overfull \hbox \(XXXpt too wide\) 
[]\T1/phv/m/n/10 Now we can do [] in the Do-cOnce source

Overfull \hbox \(XXXpt too wide\) 
[]\T1/phv/m/n/10 The bib-li-og-ra-phy is spec-i-fied by a line \T1/lmtt/m/n/10 
BIBFILE: papers.pub\T1/phv/m/n/10 , where \T1/lmtt/m/n/10 papers.pub

Overfull \hbox \(XXXpt too wide\) 
\T1/phv/m/n/10 (usu-ally af-ter the ti-tle, au-thors, and date). In this case t
he out-put text is \T1/lmtt/m/n/10 internal

Overfull \hbox \(XXXpt too wide\) 
\T1/phv/m/n/10 ment, re-spec-tively. Com-bine with [] and []

Overfull \hbox \(XXXpt too wide\) 
\T1/phv/m/n/10 the GitHub project and ex-am-ine the Do-cOnce source and the \T1
/lmtt/m/n/10 doc/src/make.sh

Overfull \hbox \(XXXpt too wide\) 
[]\T1/phv/m/n/10 Excellent "Sphinx Tu-to-rial" by C. Reller: "http://people.ee.
ethz.ch/ creller/web/tricks/reST.html" 
[24] (./quickref.aux)

 *File List*
 article.cls    2014/09/29 v1.4h Standard LaTeX document class
  size10.clo    2014/09/29 v1.4h Standard LaTeX file (size option)
 relsize.sty    2013/03/29 ver 4.1
 makeidx.sty    2014/09/29 v1.0m Standard LaTeX package
   color.sty    1999/02/16
   color.cfg    2016/01/02 v1.6 sample color configuration
  pdftex.def    2018/01/08 v1.0l Graphics/color driver for pdftex
setspace.sty    2011/12/19 v6.7a set line spacing
 amsmath.sty    2017/09/02 v2.17a AMS math features
 amstext.sty    2000/06/29 v2.01 AMS text
  amsgen.sty    1999/11/30 v2.0 generic functions
  amsbsy.sty    1999/11/29 v1.2d Bold Symbols
  amsopn.sty    2016/03/08 v2.02 operator names
amsfonts.sty    2013/01/14 v3.01 Basic AMSFonts support
 amssymb.sty    2013/01/14 v3.01 AMS font symbols
  xcolor.sty    2016/05/11 v2.12 LaTeX color extensions (UK)
   color.cfg    2016/01/02 v1.6 sample color configuration
colortbl.sty    2012/02/13 v1.0a Color table columns (DPC)
   array.sty    2016/10/06 v2.4d Tabular extension package (FMi)
      bm.sty    2017/01/16 v1.2c Bold Symbol Support (DPC/FMi)
 ltablex.sty    2014/08/13 v1.1 Modified tabularx
longtable.sty    2014/10/28 v4.11 Multi-page Table package (DPC)
tabularx.sty    2016/02/03 v2.11 `tabularx' package (DPC)
microtype.sty    2018/01/14 v2.7a Micro-typographical refinements (RS)
  keyval.sty    2014/10/28 v1.15 key=value parser (DPC)
microtype-pdftex.def    2018/01/14 v2.7a Definitions specific to pdftex (RS)
microtype.cfg    2018/01/14 v2.7a microtype main configuration file (RS)
graphicx.sty    2017/06/01 v1.1a Enhanced LaTeX Graphics (DPC,SPQR)
graphics.sty    2017/06/25 v1.2c Standard LaTeX Graphics (DPC,SPQR)
    trig.sty    2016/01/03 v1.10 sin cos tan (DPC)
graphics.cfg    2016/06/04 v1.11 sample graphics configuration
    soul.sty    2003/11/17 v2.4 letterspacing/underlining (mf)
  framed.sty    2011/10/22 v 0.96: framed or shaded text with page breaks
moreverb.sty    2008/06/03 v2.3a `more' verbatim facilities
verbatim.sty    2014/10/28 v1.5q LaTeX2e package for verbatim enhancements
 fontenc.sty
   t1enc.def    2017/04/05 v2.0i Standard LaTeX file
     ucs.sty    2013/05/11 v2.2 UCS: Unicode input support
uni-global.def    2013/05/13 UCS: Unicode global data
inputenc.sty    2015/03/17 v1.2c Input encoding file
   utf8x.def    2004/10/17 UCS: Input encoding UTF-8
  helvet.sty    2005/04/12 PSNFSS-v9.2a (WaS) 
 lmodern.sty    2009/10/30 v1.6 Latin Modern Fonts
hyperref.sty    2018/02/06 v6.86b Hypertext links for LaTeX
hobsub-hyperref.sty    2016/05/16 v1.14 Bundle oberdiek, subset hyperref (HO)
hobsub-generic.sty    2016/05/16 v1.14 Bundle oberdiek, subset generic (HO)
  hobsub.sty    2016/05/16 v1.14 Construct package bundles (HO)
infwarerr.sty    2016/05/16 v1.4 Providing info/warning/error messages (HO)
 ltxcmds.sty    2016/05/16 v1.23 LaTeX kernel commands for general use (HO)
ifluatex.sty    2016/05/16 v1.4 Provides the ifluatex switch (HO)
  ifvtex.sty    2016/05/16 v1.6 Detect VTeX and its facilities (HO)
 intcalc.sty    2016/05/16 v1.2 Expandable calculations with integers (HO)
   ifpdf.sty    2017/03/15 v3.2 Provides the ifpdf switch
etexcmds.sty    2016/05/16 v1.6 Avoid name clashes with e-TeX commands (HO)
kvsetkeys.sty    2016/05/16 v1.17 Key value parser (HO)
kvdefinekeys.sty    2016/05/16 v1.4 Define keys (HO)
pdftexcmds.sty    2018/01/21 v0.26 Utility functions of pdfTeX for LuaTeX (HO)
pdfescape.sty    2016/05/16 v1.14 Implements pdfTeX's escape features (HO)
bigintcalc.sty    2016/05/16 v1.4 Expandable calculations on big integers (HO)
  bitset.sty    2016/05/16 v1.2 Handle bit-vector datatype (HO)
uniquecounter.sty    2016/05/16 v1.3 Provide unlimited unique counter (HO)
letltxmacro.sty    2016/05/16 v1.5 Let assignment for LaTeX macros (HO)
 hopatch.sty    2016/05/16 v1.3 Wrapper for package hooks (HO)
xcolor-patch.sty    2016/05/16 xcolor patch
atveryend.sty    2016/05/16 v1.9 Hooks at the very end of document (HO)
atbegshi.sty    2016/06/09 v1.18 At begin shipout hook (HO)
refcount.sty    2016/05/16 v3.5 Data extraction from label references (HO)
 hycolor.sty    2016/05/16 v1.8 Color options for hyperref/bookmark (HO)
 ifxetex.sty    2010/09/12 v0.6 Provides ifxetex conditional
 auxhook.sty    2016/05/16 v1.4 Hooks for auxiliary files (HO)
kvoptions.sty    2016/05/16 v3.12 Key value format for package options (HO)
  pd1enc.def    2018/02/06 v6.86b Hyperref: PDFDocEncoding definition (HO)
hyperref.cfg    2002/06/06 v1.2 hyperref configuration of TeXLive
     url.sty    2013/09/16  ver 3.4  Verb mode for urls, etc.
 hpdftex.def    2018/02/06 v6.86b Hyperref driver for pdfTeX
rerunfilecheck.sty    2016/05/16 v1.8 Rerun checks for auxiliary files (HO)
placeins.sty    2005/04/18  v 2.2
mdframed.sty    2014/05/30 2.0: mdframed
  xparse.sty    2018/02/21 L3 Experimental document command parser
   expl3.sty    2018/02/21 L3 programming layer (loader) 
expl3-code.tex    2018/02/21 L3 programming layer 
l3pdfmode.def    2017/03/18 v L3 Experimental driver: PDF mode
etoolbox.sty    2018/02/11 v2.5e e-TeX tools for LaTeX (JAW)
zref-abspage.sty    2016/05/21 v2.26 Module abspage for zref (HO)
zref-base.sty    2016/05/21 v2.26 Module base for zref (HO)
needspace.sty    2010/09/12 v1.3d reserve vertical space
    tikz.sty    2015/08/07 v3.0.1a (rcs-revision 1.151)
     pgf.sty    2015/08/07 v3.0.1a (rcs-revision 1.15)
  pgfrcs.sty    2015/08/07 v3.0.1a (rcs-revision 1.31)
everyshi.sty    2001/05/15 v3.00 EveryShipout Package (MS)
  pgfrcs.code.tex
 pgfcore.sty    2010/04/11 v3.0.1a (rcs-revision 1.7)
  pgfsys.sty    2014/07/09 v3.0.1a (rcs-revision 1.48)
  pgfsys.code.tex
pgfsyssoftpath.code.tex    2013/09/09  (rcs-revision 1.9)
pgfsysprotocol.code.tex    2006/10/16  (rcs-revision 1.4)
 pgfcore.code.tex
pgfcomp-version-0-65.sty    2007/07/03 v3.0.1a (rcs-revision 1.7)
pgfcomp-version-1-18.sty    2007/07/23 v3.0.1a (rcs-revision 1.1)
  pgffor.sty    2013/12/13 v3.0.1a (rcs-revision 1.25)
 pgfkeys.sty    
 pgfkeys.code.tex
 pgfmath.sty    
 pgfmath.code.tex
  pgffor.code.tex
    tikz.code.tex
md-frame-1.mdf    2014/05/30\ 2.0: md-frame-1
    calc.sty    2014/10/28 v4.3 Infix arithmetic (KKT,FJ)
idxlayout.sty    2012/03/30 v0.4d Configurable index layout
multicol.sty    2017/04/11 v1.8q multicolumn formatting (FMi)
tocbibind.sty    2010/10/13 v1.5k extra ToC listings
ragged2e.sty    2009/05/21 v2.1 ragged2e Package (MS)
everysel.sty    2011/10/28 v1.2 EverySelectfont Package (MS)
   t1phv.fd    2001/06/04 scalable font definitions for T1/phv.
supp-pdf.mkii
epstopdf-base.sty    2016/05/15 v2.6 Base part for package epstopdf
  grfext.sty    2016/05/16 v1.2 Manage graphics extensions (HO)
epstopdf-sys.cfg    2010/07/13 v1.3 Configuration of (r)epstopdf for TeX Live
 ucsencs.def    2011/01/21 Fixes to fontencodings LGR, T3
 nameref.sty    2016/05/21 v2.44 Cross-referencing by name of section
gettitlestring.sty    2016/05/16 v1.5 Cleanup title references (HO)
  ot1lmr.fd    2009/10/30 v1.6 Font defs for Latin Modern
  mt-cmr.cfg    2013/05/19 v2.2 microtype config. file: Computer Modern Roman (
RS)
  omllmm.fd    2009/10/30 v1.6 Font defs for Latin Modern
 omslmsy.fd    2009/10/30 v1.6 Font defs for Latin Modern
 omxlmex.fd    2009/10/30 v1.6 Font defs for Latin Modern
    umsa.fd    2013/01/14 v3.01 AMS symbols A
  mt-msa.cfg    2006/02/04 v1.1 microtype config. file: AMS symbols (a) (RS)
    umsb.fd    2013/01/14 v3.01 AMS symbols B
  mt-msb.cfg    2005/06/01 v1.0 microtype config. file: AMS symbols (b) (RS)
  t1lmtt.fd    2009/10/30 v1.6 Font defs for Latin Modern
  omsphv.fd    
latex_figs/dizzy_face.png
 ***********


Package rerunfilecheck Warning: File `quickref.out' has changed.
(rerunfilecheck)                Rerun to get outlines right
(rerunfilecheck)                or use package `bookmark'.


LaTeX Warning: There were undefined references.


LaTeX Warning: Label(s) may have changed. Rerun to get cross-references right.

 )
(see the transcript file for additional information){/usr/share/texmf/fonts/enc
/dvips/lm/lm-ec.enc}{/usr/share/texmf/fonts/enc/dvips/lm/lm-mathsy.enc}{/usr/sh
are/texmf/fonts/enc/dvips/lm/lm-rm.enc}{/usr/share/texmf/fonts/enc/dvips/lm/lm-
mathit.enc}{/usr/share/texlive/texmf-dist/fonts/enc/dvips/base/8r.enc}</usr/sha
re/texlive/texmf-dist/fonts/type1/public/amsfonts/cm/cmsy10.pfb></usr/share/tex
mf/fonts/type1/public/lm/lmmi10.pfb></usr/share/texmf/fonts/type1/public/lm/lmm
i7.pfb></usr/share/texmf/fonts/type1/public/lm/lmr10.pfb></usr/share/texmf/font
s/type1/public/lm/lmr6.pfb></usr/share/texmf/fonts/type1/public/lm/lmr7.pfb></u
sr/share/texmf/fonts/type1/public/lm/lmsy10.pfb></usr/share/texmf/fonts/type1/p
ublic/lm/lmtk10.pfb></usr/share/texmf/fonts/type1/public/lm/lmtt10.pfb></usr/sh
are/texmf/fonts/type1/public/lm/lmtt8.pfb></usr/share/texmf/fonts/type1/public/
lm/lmtt9.pfb></usr/share/texlive/texmf-dist/fonts/type1/urw/helvetic/uhvb8a.pfb
></usr/share/texlive/texmf-dist/fonts/type1/urw/helvetic/uhvr8a.pfb></usr/share
/texlive/texmf-dist/fonts/type1/urw/helvetic/uhvro8a.pfb>
Output written on quickref.pdf (XXX pages, ).
Transcript written on quickref.log.
+ '[' 0 -ne 0 ']'
+ system pdflatex -shell-escape quickref
+ pdflatex -shell-escape quickref
This is pdfTeX, Version 3.14159265-2.6-1.40.18 (TeX Live 2017/Debian) (preloaded format=pdflatex)
 \write18 enabled.
entering extended mode
(./quickref.tex
LaTeX2e <2017-04-15>
Babel <3.18> and hyphenation patterns for 84 language(s) loaded.
(/usr/share/texlive/texmf-dist/tex/latex/base/article.cls
Document Class: article 2014/09/29 v1.4h Standard LaTeX document class



(/usr/share/texlive/texmf-dist/tex/latex/graphics/color.sty



(/usr/share/texlive/texmf-dist/tex/latex/amsmath/amsmath.sty
For additional information on amsmath, use the `?' option.
(/usr/share/texlive/texmf-dist/tex/latex/amsmath/amstext.sty





(/usr/share/texlive/texmf-dist/tex/latex/xcolor/xcolor.sty

(/usr/share/texlive/texmf-dist/tex/latex/colortbl/colortbl.sty


(/usr/share/texlive/texmf-dist/tex/latex/ltablex/ltablex.sty


(/usr/share/texlive/texmf-dist/tex/latex/microtype/microtype.sty



(/usr/share/texlive/texmf-dist/tex/latex/graphics/graphicx.sty
(/usr/share/texlive/texmf-dist/tex/latex/graphics/graphics.sty



(/usr/share/texlive/texmf-dist/tex/latex/fancyvrb/fancyvrb.sty
Style option: `fancyvrb' v2.7a, with DG/SPQR fixes, and firstline=lastline fix 
<2008/02/07> (tvz)) 
 (/usr/share/texlive/texmf-dist/tex/latex/moreverb/moreverb.sty

(/usr/share/texlive/texmf-dist/tex/latex/base/fontenc.sty

(/usr/share/texlive/texmf-dist/tex/latex/ucs/ucs.sty

(/usr/share/texlive/texmf-dist/tex/latex/base/inputenc.sty



(/usr/share/texlive/texmf-dist/tex/latex/hyperref/hyperref.sty
(/usr/share/texlive/texmf-dist/tex/generic/oberdiek/hobsub-hyperref.sty







(/usr/share/texlive/texmf-dist/tex/latex/hyperref/hpdftex.def


(/X/X/mdframed.sty
(/usr/share/texlive/texmf-dist/tex/latex/l3packages/xparse/xparse.sty
(/usr/share/texlive/texmf-dist/tex/latex/l3kernel/expl3.sty



(/usr/share/texlive/texmf-dist/tex/latex/oberdiek/zref-abspage.sty


(/usr/share/texlive/texmf-dist/tex/latex/pgf/frontendlayer/tikz.sty
(/usr/share/texlive/texmf-dist/tex/latex/pgf/basiclayer/pgf.sty
(/usr/share/texlive/texmf-dist/tex/latex/pgf/utilities/pgfrcs.sty
(/usr/share/texlive/texmf-dist/tex/generic/pgf/utilities/pgfutil-common.tex
(/usr/share/texlive/texmf-dist/tex/generic/pgf/utilities/pgfutil-common-lists.t
ex)) (/usr/share/texlive/texmf-dist/tex/generic/pgf/utilities/pgfutil-latex.def


(/usr/share/texlive/texmf-dist/tex/latex/pgf/basiclayer/pgfcore.sty
(/usr/share/texlive/texmf-dist/tex/latex/pgf/systemlayer/pgfsys.sty
(/usr/share/texlive/texmf-dist/tex/generic/pgf/systemlayer/pgfsys.code.tex
(/usr/share/texlive/texmf-dist/tex/generic/pgf/utilities/pgfkeys.code.tex
(/usr/share/texlive/texmf-dist/tex/generic/pgf/utilities/pgfkeysfiltered.code.t
ex)) 
(/usr/share/texlive/texmf-dist/tex/generic/pgf/systemlayer/pgfsys-pdftex.def
(/usr/share/texlive/texmf-dist/tex/generic/pgf/systemlayer/pgfsys-common-pdf.de
f)))
(/usr/share/texlive/texmf-dist/tex/generic/pgf/systemlayer/pgfsyssoftpath.code.
tex)
(/usr/share/texlive/texmf-dist/tex/generic/pgf/systemlayer/pgfsysprotocol.code.
tex))
(/usr/share/texlive/texmf-dist/tex/generic/pgf/basiclayer/pgfcore.code.tex
(/usr/share/texlive/texmf-dist/tex/generic/pgf/math/pgfmath.code.tex
(/usr/share/texlive/texmf-dist/tex/generic/pgf/math/pgfmathcalc.code.tex


(/usr/share/texlive/texmf-dist/tex/generic/pgf/math/pgfmathfunctions.code.tex
(/usr/share/texlive/texmf-dist/tex/generic/pgf/math/pgfmathfunctions.basic.code
.tex)
(/usr/share/texlive/texmf-dist/tex/generic/pgf/math/pgfmathfunctions.trigonomet
ric.code.tex)
(/usr/share/texlive/texmf-dist/tex/generic/pgf/math/pgfmathfunctions.random.cod
e.tex)
(/usr/share/texlive/texmf-dist/tex/generic/pgf/math/pgfmathfunctions.comparison
.code.tex)
(/usr/share/texlive/texmf-dist/tex/generic/pgf/math/pgfmathfunctions.base.code.
tex)
(/usr/share/texlive/texmf-dist/tex/generic/pgf/math/pgfmathfunctions.round.code
.tex)
(/usr/share/texlive/texmf-dist/tex/generic/pgf/math/pgfmathfunctions.misc.code.
tex)
(/usr/share/texlive/texmf-dist/tex/generic/pgf/math/pgfmathfunctions.integerari
thmetics.code.tex)))

(/usr/share/texlive/texmf-dist/tex/generic/pgf/basiclayer/pgfcorepoints.code.te
x)
(/usr/share/texlive/texmf-dist/tex/generic/pgf/basiclayer/pgfcorepathconstruct.
code.tex)
(/usr/share/texlive/texmf-dist/tex/generic/pgf/basiclayer/pgfcorepathusage.code
.tex)
(/usr/share/texlive/texmf-dist/tex/generic/pgf/basiclayer/pgfcorescopes.code.te
x)
(/usr/share/texlive/texmf-dist/tex/generic/pgf/basiclayer/pgfcoregraphicstate.c
ode.tex)
(/usr/share/texlive/texmf-dist/tex/generic/pgf/basiclayer/pgfcoretransformation
s.code.tex)
(/usr/share/texlive/texmf-dist/tex/generic/pgf/basiclayer/pgfcorequick.code.tex
)
(/usr/share/texlive/texmf-dist/tex/generic/pgf/basiclayer/pgfcoreobjects.code.t
ex)
(/usr/share/texlive/texmf-dist/tex/generic/pgf/basiclayer/pgfcorepathprocessing
.code.tex)
(/usr/share/texlive/texmf-dist/tex/generic/pgf/basiclayer/pgfcorearrows.code.te
x)
(/usr/share/texlive/texmf-dist/tex/generic/pgf/basiclayer/pgfcoreshade.code.tex
)
(/usr/share/texlive/texmf-dist/tex/generic/pgf/basiclayer/pgfcoreimage.code.tex

(/usr/share/texlive/texmf-dist/tex/generic/pgf/basiclayer/pgfcoreexternal.code.
tex))
(/usr/share/texlive/texmf-dist/tex/generic/pgf/basiclayer/pgfcorelayers.code.te
x)
(/usr/share/texlive/texmf-dist/tex/generic/pgf/basiclayer/pgfcoretransparency.c
ode.tex)
(/usr/share/texlive/texmf-dist/tex/generic/pgf/basiclayer/pgfcorepatterns.code.
tex)))
(/usr/share/texlive/texmf-dist/tex/generic/pgf/modules/pgfmoduleshapes.code.tex
) (/usr/share/texlive/texmf-dist/tex/generic/pgf/modules/pgfmoduleplot.code.tex
)
(/usr/share/texlive/texmf-dist/tex/latex/pgf/compatibility/pgfcomp-version-0-65
.sty)
(/usr/share/texlive/texmf-dist/tex/latex/pgf/compatibility/pgfcomp-version-1-18
.sty)) (/usr/share/texlive/texmf-dist/tex/latex/pgf/utilities/pgffor.sty
(/usr/share/texlive/texmf-dist/tex/latex/pgf/utilities/pgfkeys.sty

(/usr/share/texlive/texmf-dist/tex/latex/pgf/math/pgfmath.sty

(/usr/share/texlive/texmf-dist/tex/generic/pgf/utilities/pgffor.code.tex

(/usr/share/texlive/texmf-dist/tex/generic/pgf/frontendlayer/tikz/tikz.code.tex

(/usr/share/texlive/texmf-dist/tex/generic/pgf/libraries/pgflibraryplothandlers
.code.tex)
(/usr/share/texlive/texmf-dist/tex/generic/pgf/modules/pgfmodulematrix.code.tex
)
(/usr/share/texlive/texmf-dist/tex/generic/pgf/frontendlayer/tikz/libraries/tik
zlibrarytopaths.code.tex)))
(/X/X/md-frame-1.mdf))

Writing index file quickref.idx
(/usr/share/texlive/texmf-dist/tex/latex/idxlayout/idxlayout.sty

(/usr/share/texlive/texmf-dist/tex/latex/tocbibind/tocbibind.sty

Package tocbibind Note: Using section or other style headings.

) (/usr/share/texlive/texmf-dist/tex/latex/ms/ragged2e.sty


(/usr/share/texlive/texmf-dist/tex/context/base/mkii/supp-pdf.mkii
[Loading MPS to PDF converter (version 2006.09.02).]
) (/usr/share/texlive/texmf-dist/tex/latex/oberdiek/epstopdf-base.sty



(/usr/share/texlive/texmf-dist/tex/latex/hyperref/nameref.sty

(./quickref.out) (./quickref.out) ABD: EveryShipout initializing macros
ABD: EverySelectfont initializing macros








 (./quickref.toc
 [1{/var/lib/texmf/fonts/map/pdftex/u
pdmap/pdftex.map}] 

Overfull \hbox \(XXXpt too wide\) 
\T1/phv/m/n/10 Note that ab-stracts are rec-og-nized by start-ing with [] or []

[4] [5] [6 <./latex_figs/dizzy_face.png>] [7] [8]
Overfull \hbox \(XXXpt too wide\) 
[]\T1/phv/m/n/10 resulting in Some sen-tence with \T1/lmtt/m/n/10 words in verb
atim style\T1/phv/m/n/10 . Multi-line blocks

Overfull \hbox \(XXXpt too wide\) 
[]\T1/phv/m/n/10 that maps en-vi-ron-ments (\T1/lmtt/m/n/10 xxx\T1/phv/m/n/10 )
 onto valid lan-guage types for Pyg-ments (which

Overfull \hbox \(XXXpt too wide\) 
\T1/phv/m/n/10 ning text. New-com-mands must be de-fined in files with names \T
1/lmtt/m/n/10 newcommands*.tex\T1/phv/m/n/10 .

Overfull \hbox \(XXXpt too wide\) 
\T1/phv/m/n/10 but you can com-bine the files into one file us-ing the []

Overfull \hbox \(XXXpt too wide\) 
\T1/phv/m/n/10 tools for, e.g., com-ment-ing out/in large por-tions of text and
 cre-at-ing macros. 

Overfull \hbox \(XXXpt too wide\) 
[]\T1/phv/m/n/10 Now we can do [] in the Do-cOnce source

Overfull \hbox \(XXXpt too wide\) 
[]\T1/phv/m/n/10 The bib-li-og-ra-phy is spec-i-fied by a line \T1/lmtt/m/n/10 
BIBFILE: papers.pub\T1/phv/m/n/10 , where \T1/lmtt/m/n/10 papers.pub

Overfull \hbox \(XXXpt too wide\) 
\T1/phv/m/n/10 (usu-ally af-ter the ti-tle, au-thors, and date). In this case t
he out-put text is \T1/lmtt/m/n/10 internal

Overfull \hbox \(XXXpt too wide\) 
\T1/phv/m/n/10 ment, re-spec-tively. Com-bine with [] and []

Overfull \hbox \(XXXpt too wide\) 
\T1/phv/m/n/10 the GitHub project and ex-am-ine the Do-cOnce source and the \T1
/lmtt/m/n/10 doc/src/make.sh

Overfull \hbox \(XXXpt too wide\) 
[]\T1/phv/m/n/10 Excellent "Sphinx Tu-to-rial" by C. Reller: "http://people.ee.
ethz.ch/ creller/web/tricks/reST.html" 
[25] (./quickref.aux)

 *File List*
 article.cls    2014/09/29 v1.4h Standard LaTeX document class
  size10.clo    2014/09/29 v1.4h Standard LaTeX file (size option)
 relsize.sty    2013/03/29 ver 4.1
 makeidx.sty    2014/09/29 v1.0m Standard LaTeX package
   color.sty    1999/02/16
   color.cfg    2016/01/02 v1.6 sample color configuration
  pdftex.def    2018/01/08 v1.0l Graphics/color driver for pdftex
setspace.sty    2011/12/19 v6.7a set line spacing
 amsmath.sty    2017/09/02 v2.17a AMS math features
 amstext.sty    2000/06/29 v2.01 AMS text
  amsgen.sty    1999/11/30 v2.0 generic functions
  amsbsy.sty    1999/11/29 v1.2d Bold Symbols
  amsopn.sty    2016/03/08 v2.02 operator names
amsfonts.sty    2013/01/14 v3.01 Basic AMSFonts support
 amssymb.sty    2013/01/14 v3.01 AMS font symbols
  xcolor.sty    2016/05/11 v2.12 LaTeX color extensions (UK)
   color.cfg    2016/01/02 v1.6 sample color configuration
colortbl.sty    2012/02/13 v1.0a Color table columns (DPC)
   array.sty    2016/10/06 v2.4d Tabular extension package (FMi)
      bm.sty    2017/01/16 v1.2c Bold Symbol Support (DPC/FMi)
 ltablex.sty    2014/08/13 v1.1 Modified tabularx
longtable.sty    2014/10/28 v4.11 Multi-page Table package (DPC)
tabularx.sty    2016/02/03 v2.11 `tabularx' package (DPC)
microtype.sty    2018/01/14 v2.7a Micro-typographical refinements (RS)
  keyval.sty    2014/10/28 v1.15 key=value parser (DPC)
microtype-pdftex.def    2018/01/14 v2.7a Definitions specific to pdftex (RS)
microtype.cfg    2018/01/14 v2.7a microtype main configuration file (RS)
graphicx.sty    2017/06/01 v1.1a Enhanced LaTeX Graphics (DPC,SPQR)
graphics.sty    2017/06/25 v1.2c Standard LaTeX Graphics (DPC,SPQR)
    trig.sty    2016/01/03 v1.10 sin cos tan (DPC)
graphics.cfg    2016/06/04 v1.11 sample graphics configuration
    soul.sty    2003/11/17 v2.4 letterspacing/underlining (mf)
  framed.sty    2011/10/22 v 0.96: framed or shaded text with page breaks
moreverb.sty    2008/06/03 v2.3a `more' verbatim facilities
verbatim.sty    2014/10/28 v1.5q LaTeX2e package for verbatim enhancements
 fontenc.sty
   t1enc.def    2017/04/05 v2.0i Standard LaTeX file
     ucs.sty    2013/05/11 v2.2 UCS: Unicode input support
uni-global.def    2013/05/13 UCS: Unicode global data
inputenc.sty    2015/03/17 v1.2c Input encoding file
   utf8x.def    2004/10/17 UCS: Input encoding UTF-8
  helvet.sty    2005/04/12 PSNFSS-v9.2a (WaS) 
 lmodern.sty    2009/10/30 v1.6 Latin Modern Fonts
hyperref.sty    2018/02/06 v6.86b Hypertext links for LaTeX
hobsub-hyperref.sty    2016/05/16 v1.14 Bundle oberdiek, subset hyperref (HO)
hobsub-generic.sty    2016/05/16 v1.14 Bundle oberdiek, subset generic (HO)
  hobsub.sty    2016/05/16 v1.14 Construct package bundles (HO)
infwarerr.sty    2016/05/16 v1.4 Providing info/warning/error messages (HO)
 ltxcmds.sty    2016/05/16 v1.23 LaTeX kernel commands for general use (HO)
ifluatex.sty    2016/05/16 v1.4 Provides the ifluatex switch (HO)
  ifvtex.sty    2016/05/16 v1.6 Detect VTeX and its facilities (HO)
 intcalc.sty    2016/05/16 v1.2 Expandable calculations with integers (HO)
   ifpdf.sty    2017/03/15 v3.2 Provides the ifpdf switch
etexcmds.sty    2016/05/16 v1.6 Avoid name clashes with e-TeX commands (HO)
kvsetkeys.sty    2016/05/16 v1.17 Key value parser (HO)
kvdefinekeys.sty    2016/05/16 v1.4 Define keys (HO)
pdftexcmds.sty    2018/01/21 v0.26 Utility functions of pdfTeX for LuaTeX (HO)
pdfescape.sty    2016/05/16 v1.14 Implements pdfTeX's escape features (HO)
bigintcalc.sty    2016/05/16 v1.4 Expandable calculations on big integers (HO)
  bitset.sty    2016/05/16 v1.2 Handle bit-vector datatype (HO)
uniquecounter.sty    2016/05/16 v1.3 Provide unlimited unique counter (HO)
letltxmacro.sty    2016/05/16 v1.5 Let assignment for LaTeX macros (HO)
 hopatch.sty    2016/05/16 v1.3 Wrapper for package hooks (HO)
xcolor-patch.sty    2016/05/16 xcolor patch
atveryend.sty    2016/05/16 v1.9 Hooks at the very end of document (HO)
atbegshi.sty    2016/06/09 v1.18 At begin shipout hook (HO)
refcount.sty    2016/05/16 v3.5 Data extraction from label references (HO)
 hycolor.sty    2016/05/16 v1.8 Color options for hyperref/bookmark (HO)
 ifxetex.sty    2010/09/12 v0.6 Provides ifxetex conditional
 auxhook.sty    2016/05/16 v1.4 Hooks for auxiliary files (HO)
kvoptions.sty    2016/05/16 v3.12 Key value format for package options (HO)
  pd1enc.def    2018/02/06 v6.86b Hyperref: PDFDocEncoding definition (HO)
hyperref.cfg    2002/06/06 v1.2 hyperref configuration of TeXLive
     url.sty    2013/09/16  ver 3.4  Verb mode for urls, etc.
 hpdftex.def    2018/02/06 v6.86b Hyperref driver for pdfTeX
rerunfilecheck.sty    2016/05/16 v1.8 Rerun checks for auxiliary files (HO)
placeins.sty    2005/04/18  v 2.2
mdframed.sty    2014/05/30 2.0: mdframed
  xparse.sty    2018/02/21 L3 Experimental document command parser
   expl3.sty    2018/02/21 L3 programming layer (loader) 
expl3-code.tex    2018/02/21 L3 programming layer 
l3pdfmode.def    2017/03/18 v L3 Experimental driver: PDF mode
etoolbox.sty    2018/02/11 v2.5e e-TeX tools for LaTeX (JAW)
zref-abspage.sty    2016/05/21 v2.26 Module abspage for zref (HO)
zref-base.sty    2016/05/21 v2.26 Module base for zref (HO)
needspace.sty    2010/09/12 v1.3d reserve vertical space
    tikz.sty    2015/08/07 v3.0.1a (rcs-revision 1.151)
     pgf.sty    2015/08/07 v3.0.1a (rcs-revision 1.15)
  pgfrcs.sty    2015/08/07 v3.0.1a (rcs-revision 1.31)
everyshi.sty    2001/05/15 v3.00 EveryShipout Package (MS)
  pgfrcs.code.tex
 pgfcore.sty    2010/04/11 v3.0.1a (rcs-revision 1.7)
  pgfsys.sty    2014/07/09 v3.0.1a (rcs-revision 1.48)
  pgfsys.code.tex
pgfsyssoftpath.code.tex    2013/09/09  (rcs-revision 1.9)
pgfsysprotocol.code.tex    2006/10/16  (rcs-revision 1.4)
 pgfcore.code.tex
pgfcomp-version-0-65.sty    2007/07/03 v3.0.1a (rcs-revision 1.7)
pgfcomp-version-1-18.sty    2007/07/23 v3.0.1a (rcs-revision 1.1)
  pgffor.sty    2013/12/13 v3.0.1a (rcs-revision 1.25)
 pgfkeys.sty    
 pgfkeys.code.tex
 pgfmath.sty    
 pgfmath.code.tex
  pgffor.code.tex
    tikz.code.tex
md-frame-1.mdf    2014/05/30\ 2.0: md-frame-1
    calc.sty    2014/10/28 v4.3 Infix arithmetic (KKT,FJ)
idxlayout.sty    2012/03/30 v0.4d Configurable index layout
multicol.sty    2017/04/11 v1.8q multicolumn formatting (FMi)
tocbibind.sty    2010/10/13 v1.5k extra ToC listings
ragged2e.sty    2009/05/21 v2.1 ragged2e Package (MS)
everysel.sty    2011/10/28 v1.2 EverySelectfont Package (MS)
   t1phv.fd    2001/06/04 scalable font definitions for T1/phv.
supp-pdf.mkii
epstopdf-base.sty    2016/05/15 v2.6 Base part for package epstopdf
  grfext.sty    2016/05/16 v1.2 Manage graphics extensions (HO)
epstopdf-sys.cfg    2010/07/13 v1.3 Configuration of (r)epstopdf for TeX Live
 ucsencs.def    2011/01/21 Fixes to fontencodings LGR, T3
 nameref.sty    2016/05/21 v2.44 Cross-referencing by name of section
gettitlestring.sty    2016/05/16 v1.5 Cleanup title references (HO)
quickref.out
quickref.out
  ot1lmr.fd    2009/10/30 v1.6 Font defs for Latin Modern
  mt-cmr.cfg    2013/05/19 v2.2 microtype config. file: Computer Modern Roman (
RS)
  omllmm.fd    2009/10/30 v1.6 Font defs for Latin Modern
 omslmsy.fd    2009/10/30 v1.6 Font defs for Latin Modern
 omxlmex.fd    2009/10/30 v1.6 Font defs for Latin Modern
    umsa.fd    2013/01/14 v3.01 AMS symbols A
  mt-msa.cfg    2006/02/04 v1.1 microtype config. file: AMS symbols (a) (RS)
    umsb.fd    2013/01/14 v3.01 AMS symbols B
  mt-msb.cfg    2005/06/01 v1.0 microtype config. file: AMS symbols (b) (RS)
  t1lmtt.fd    2009/10/30 v1.6 Font defs for Latin Modern
  omsphv.fd    
latex_figs/dizzy_face.png
 ***********


LaTeX Warning: Label(s) may have changed. Rerun to get cross-references right.

 )
(see the transcript file for additional information){/usr/share/texmf/fonts/enc
/dvips/lm/lm-ec.enc}{/usr/share/texmf/fonts/enc/dvips/lm/lm-mathsy.enc}{/usr/sh
are/texmf/fonts/enc/dvips/lm/lm-rm.enc}{/usr/share/texmf/fonts/enc/dvips/lm/lm-
mathit.enc}{/usr/share/texlive/texmf-dist/fonts/enc/dvips/base/8r.enc}</usr/sha
re/texlive/texmf-dist/fonts/type1/public/amsfonts/cm/cmsy10.pfb></usr/share/tex
mf/fonts/type1/public/lm/lmmi10.pfb></usr/share/texmf/fonts/type1/public/lm/lmm
i7.pfb></usr/share/texmf/fonts/type1/public/lm/lmr10.pfb></usr/share/texmf/font
s/type1/public/lm/lmr6.pfb></usr/share/texmf/fonts/type1/public/lm/lmr7.pfb></u
sr/share/texmf/fonts/type1/public/lm/lmsy10.pfb></usr/share/texmf/fonts/type1/p
ublic/lm/lmtk10.pfb></usr/share/texmf/fonts/type1/public/lm/lmtt10.pfb></usr/sh
are/texmf/fonts/type1/public/lm/lmtt8.pfb></usr/share/texmf/fonts/type1/public/
lm/lmtt9.pfb></usr/share/texlive/texmf-dist/fonts/type1/urw/helvetic/uhvb8a.pfb
></usr/share/texlive/texmf-dist/fonts/type1/urw/helvetic/uhvr8a.pfb></usr/share
/texlive/texmf-dist/fonts/type1/urw/helvetic/uhvro8a.pfb>
Output written on quickref.pdf (XXX pages, ).
Transcript written on quickref.log.
+ '[' 0 -ne 0 ']'
+ system doconce format sphinx quickref --no_preprocess --no_abort
+ doconce format sphinx quickref --no_preprocess --no_abort
running mako on quickref.do.txt to make tmp_mako__quickref.do.txt
Translating doconce text in tmp_mako__quickref.do.txt to sphinx
copy complete file doconce_program.sh  (format: shpro)
*** warning: sphinx/rst is a suboptimal format for
    typesetting edit markup such as
    [add 2: ,]
    Use HTML or LaTeX output instead, implement the
    edits (doconce apply_edit_comments) and then use sphinx.
output in quickref.rst
+ '[' 0 -ne 0 ']'
+ rm -rf sphinx-rootdir
+ touch conf.py
+ system doconce sphinx_dir theme=cbc quickref dirname=sphinx-rootdir
+ doconce sphinx_dir theme=cbc quickref dirname=sphinx-rootdir
Making sphinx-rootdir
Welcome to the Sphinx 3.3.1 quickstart utility.

Please enter values for the following settings (just press Enter to
accept a default value, if one is given in brackets).

Selected root path: .

Error: an existing conf.py has been found in the selected root path.
sphinx-quickstart will not overwrite existing Sphinx projects.

> Please enter a new root path (or just Enter to exit) []: 
You have two options for placing the build directory for Sphinx output.
Either, you use a directory "_build" within the root path, or you separate
"source" and "build" directories within the root path.
> Separate source and build directories (y/n) [n]: 
The project name will occur in several places in the built documentation.
> Project name: > Author name(s): > Project release []: 
If the documents are to be written in a language other than English,
you can select a language here by its language code. Sphinx will then
translate text that it generates into that language.

For a list of supported codes, see
https://www.sphinx-doc.org/en/master/usage/configuration.html#confval-language.
> Project language [en]: 
Creating file /X/X/sphinx-rootdir/conf.py.
Creating file /X/X/sphinx-rootdir/index.rst.
Creating file /X/X/sphinx-rootdir/Makefile.
Creating file /X/X/sphinx-rootdir/make.bat.

Finished: An initial directory structure has been created.

You should now populate your master file /X/X/sphinx-rootdir/index.rst and create other documentation
source files. Use the Makefile to build the docs, like so:
   make builder
where "builder" is one of the supported builders, e.g. html, latex or linkcheck.

title: DocOnce Quick Reference
author: Hans Petter Langtangen, H. P. Langtangen, Kaare Dump and A. Dummy Author
copyright: 2XXX, Hans Petter Langtangen, H. P. Langtangen, Kaare Dump and A. Dummy Author
theme: cbc

These Sphinx themes were found: ADCtheme, agni, agogo, alabaster, basic, basicstrap, bizstyle, bloodish, bootstrap, cbc, classic, cloud, default, epub, fenics, fenics_classic, fenics_minimal1, fenics_minimal2, haiku, jal, nature, pylons, pyramid, redcloud, scipy_lectures, scrolls, slim-agogo, solarized, sphinx_rtd_theme, sphinxdoc, traditional, uio, uio2, vlinux-theme
'automake_sphinx.py' contains the steps to (re)compile the sphinx
version. You may want to edit this file, or run the steps manually,
or just run it by

  python automake_sphinx.py

+ '[' 0 -ne 0 ']'
+ doconce replace 'doconce format sphinx %s' 'doconce format sphinx %s --no-preprocess' automake_sphinx.py
replacing doconce format sphinx %s by doconce format sphinx %s --no-preprocess in automake_sphinx.py
+ system python automake_sphinx.py
+ python automake_sphinx.py
Removing everything under '_build'...
Running Sphinx v3.3.1
loading translations [1.0]... not available for built-in messages
making output directory... done
building [mo]: targets for 0 po files that are out of date
building [html]: targets for 2 source files that are out of date
updating environment: [new config] 2 added, 0 changed, 0 removed
reading sources... [ 50%] index
reading sources... [100%] quickref

looking for now-outdated files... none found
pickling environment... done
checking consistency... done
preparing documents... done
writing output... [ 50%] index
writing output... [100%] quickref

generating indices... genindex done
writing additional pages... search done
copying static files... done
copying extra files... done
dumping search index in English (code: en)... done
dumping object inventory... done
build succeeded.

The HTML pages are in _build/html.
/X/X/sphinx-rootdir
running make clean
running make html
Fix generated files:
genindex.html
index.html
quickref.html
search.html


google-chrome sphinx-rootdir/_build/html/index.html

+ '[' 0 -ne 0 ']'
+ cp quickref.rst quickref.sphinx.rst
+ system doconce format rst quickref --no_preprocess --no_abort
+ doconce format rst quickref --no_preprocess --no_abort
running mako on quickref.do.txt to make tmp_mako__quickref.do.txt
Translating doconce text in tmp_mako__quickref.do.txt to rst
copy complete file doconce_program.sh  (format: shpro)
output in quickref.rst
+ '[' 0 -ne 0 ']'
+ rst2xml.py quickref.rst
+ rst2odt.py quickref.rst
+ rst2html.py quickref.rst
+ rst2latex.py quickref.rst
+ system latex quickref.rst.tex
+ latex quickref.rst.tex
This is pdfTeX, Version 3.14159265-2.6-1.40.18 (TeX Live 2017/Debian) (preloaded format=latex)
 restricted \write18 enabled.
entering extended mode
(./quickref.rst.tex
LaTeX2e <2017-04-15>
Babel <3.18> and hyphenation patterns for 84 language(s) loaded.
(/usr/share/texlive/texmf-dist/tex/latex/base/article.cls
Document Class: article 2014/09/29 v1.4h Standard LaTeX document class

(/usr/share/texlive/texmf-dist/tex/latex/cmap/cmap.sty

Package cmap Warning: pdftex in DVI mode - exiting.

) 
(/usr/share/texlive/texmf-dist/tex/latex/base/fontenc.sty

(/usr/share/texlive/texmf-dist/tex/latex/base/inputenc.sty
(/usr/share/texlive/texmf-dist/tex/latex/base/utf8.def









(/usr/share/texlive/texmf-dist/tex/latex/psnfss/helvet.sty


(/usr/share/texlive/texmf-dist/tex/latex/hyperref/hyperref.sty
(/usr/share/texlive/texmf-dist/tex/generic/oberdiek/hobsub-hyperref.sty







(/usr/share/texlive/texmf-dist/tex/latex/hyperref/hdvips.def
(/usr/share/texlive/texmf-dist/tex/latex/hyperref/pdfmark.def

(/usr/share/texlive/texmf-dist/tex/latex/oberdiek/bookmark.sty

No file quickref.rst.aux.

(/usr/share/texlive/texmf-dist/tex/latex/graphics/color.sty



(/usr/share/texlive/texmf-dist/tex/latex/hyperref/nameref.sty


Package hyperref Warning: Rerun to get /PageLabels entry.







Package hyperref Warning: old toc file detected, not used; run LaTeX again.


Overfull \hbox \(XXXpt too wide\) 
\T1/ptm/m/n/10 HTML. Other out-lets in-clude Google's \T1/pcr/m/n/10 blogger.co
m\T1/ptm/m/n/10 , Wikipedia/Wikibooks, IPython/Jupyter

Overfull \hbox \(XXXpt too wide\) 
 []\T1/pcr/m/n/10 AUTHOR: H. P. Langtangen at Center for Biomedical Computing, 
Simula Research Laboratory & Dept. of Informatics, Univ. of Oslo[] 

Overfull \hbox \(XXXpt too wide\) 
 []\T1/pcr/m/n/10 AUTHOR: Kaare Dump Email: dump@cyb.space.com at Segfault, Cyb
erspace Inc.[] 

Overfull \hbox \(XXXpt too wide\) 
 []\T1/pcr/m/n/10 name Email: somename@adr.net at institution1 & institution2[]

Overfull \hbox \(XXXpt too wide\) 
 []\T1/pcr/m/n/10 AUTHOR: name Email: somename@adr.net {copyright,2006-present}
 at inst1[] 

Underfull \hbox (badness 10000) 
[]|\T1/pcr/m/n/10 ======= Appendix: heading

Underfull \hbox (badness 1168) 
[]|\T1/pcr/m/n/10 ===== Appendix: heading ===== \T1/ptm/m/n/10 (5

Underfull \hbox (badness 1168) 
[]|\T1/pcr/m/n/10 ===== Exercise: heading ===== \T1/ptm/m/n/10 (5

Overfull \hbox \(XXXpt too wide\) 
\T1/ptm/m/n/10 Note that ab-stracts are rec-og-nized by start-ing with \T1/pcr/
m/n/10 __Abstract.__ \T1/ptm/m/n/10 or \T1/pcr/m/n/10 __Summary.__

Overfull \hbox \(XXXpt too wide\) 
\T1/ptm/m/it/10 sized words\T1/ptm/m/n/10 . Sim-i-larly, an un-der-score sur-ro
unds words that ap-pear in bold-face: \T1/pcr/m/n/10 _boldface_

Overfull \hbox \(XXXpt too wide\) 
 []\T1/pcr/m/n/10 This distance corresponds to 7.5~km, which is traveled in $7.
5/5$~s.[] 

Overfull \hbox \(XXXpt too wide\) 
 []\T1/pcr/m/n/10 The em-dash is used - without spaces - as alternative to hyph
en with[] 

Overfull \hbox \(XXXpt too wide\) 
 []\T1/pcr/m/n/10 *Premature optimization is the root of all evil.*--- Donald K
nuth.[] 

Overfull \hbox \(XXXpt too wide\) 
 []\T1/pcr/m/n/10 Note that sublists are consistently indented by one or more b
lanks as[] 

Overfull \hbox \(XXXpt too wide\) 
 []\T1/pcr/m/n/10 shown: bullets must exactly match and continuation lines must
 start[] 

Overfull \hbox \(XXXpt too wide\) 
 []\T1/pcr/m/n/10 Some running text. [hpl: There must be a space after the colo
n,[] 

Overfull \hbox \(XXXpt too wide\) 
 []\T1/pcr/m/n/10 doconce format html mydoc.do.txt --skip_inline_comments[] 

Overfull \hbox \(XXXpt too wide\) 
 []\T1/pcr/m/n/10 First consider a quantity $Q$. Without loss of generality, we
 assume[] 

Overfull \hbox \(XXXpt too wide\) 
 []\T1/pcr/m/n/10 $Q>0$. There are three, fundamental, basic property of $Q$.[]
 

Overfull \hbox \(XXXpt too wide\) 
 []\T1/pcr/m/n/10 we assume] $Q>0$. There are three[del: ,] fundamental[del: , 
basic][] 

Overfull \hbox \(XXXpt too wide\) 
 []\T1/pcr/m/n/10 [edit: property -> properties] of $Q$. [add: These are not[] 


Overfull \hbox \(XXXpt too wide\) 
\T1/ptm/m/n/10 three-color{red}{(\T1/ptm/b/n/10 del 5\T1/ptm/m/n/10 : ,}) fun-d
a-men-tal-color{red}{(\T1/ptm/b/n/10 del 6\T1/ptm/m/n/10 : , ba-sic}) \T1/ptm/b
/n/10 (**edit

Overfull \hbox \(XXXpt too wide\) 
 []\T1/pcr/m/n/10 # sphinx code-blocks: pycod=python cod=fortran cppcod=c++ sys
=console[] 

Overfull \hbox \(XXXpt too wide\) 
 []\T1/pcr/m/n/10 @@@CODE doconce_program.sh  fromto: doconce clean@^doconce sp
lit_rst[] 

Overfull \hbox \(XXXpt too wide\) 
 []\T1/pcr/m/n/10 @@@CODE doconce_program.sh  from-to: doconce clean@^doconce s
plit_rst[] 

Overfull \hbox \(XXXpt too wide\) 
 []\T1/pcr/m/n/10 @@@CODE doconce_program.sh  envir=shpro fromto: name=@[] 

Overfull \hbox \(XXXpt too wide\) 
 []\T1/pcr/m/n/10 \[ \frac{\partial\pmb{u}}{\partial t} + \pmb{u}\cdot\nabla\pm
b{u} = 0.\][] 

Overfull \hbox \(XXXpt too wide\) 
 []\T1/pcr/m/n/10 \[ \frac{\partial\pmb{u}}{\partial t} + \pmb{u}\cdot\nabla\pm
b{u} = 0.\][] 

Overfull \hbox \(XXXpt too wide\) 
[]\T1/ptm/m/n/10 If you want La-TeX math blocks to work with \T1/pcr/m/n/10 lat
ex\T1/ptm/m/n/10 , \T1/pcr/m/n/10 html\T1/ptm/m/n/10 , \T1/pcr/m/n/10 sphinx\T1
/ptm/m/n/10 ,

Overfull \hbox \(XXXpt too wide\) 
\T1/ptm/m/n/10 ments: \T1/pcr/m/n/10 \[ ... \]\T1/ptm/m/n/10 , \T1/pcr/m/n/10 e
quation*\T1/ptm/m/n/10 , \T1/pcr/m/n/10 equation\T1/ptm/m/n/10 , \T1/pcr/m/n/10
 align*\T1/ptm/m/n/10 , \T1/pcr/m/n/10 align\T1/ptm/m/n/10 .

Overfull \hbox \(XXXpt too wide\) 
\T1/pcr/m/n/10 alignat*\T1/ptm/m/n/10 , \T1/pcr/m/n/10 alignat\T1/ptm/m/n/10 . 
Other en-vi-ron-ments, such as \T1/pcr/m/n/10 split\T1/ptm/m/n/10 , \T1/pcr/m/n
/10 multiline\T1/ptm/m/n/10 ,

Overfull \hbox \(XXXpt too wide\) 
\T1/pcr/m/n/10 newcommands*.tex\T1/ptm/m/n/10 . Use \T1/pcr/m/n/10 \newcommands
 \T1/ptm/m/n/10 and not \T1/pcr/m/n/10 \def\T1/ptm/m/n/10 . Each

Package hyperref Warning: Ignoring empty anchor on .

Overfull \hbox \(XXXpt too wide\) 
[]  \T1/pcr/m/n/10 \includegraphics[width=0.55\linewidth]{figs/myfig.pdf}  

Overfull \hbox \(XXXpt too wide\) 
[]\T1/pcr/m/n/10 \multicolumn{1}{c}{$v_0$} & \multicolumn{1}{c}{$f_R(v_0)$}\\hl
ine  

Overfull \hbox \(XXXpt too wide\) 
 []\T1/pcr/m/n/10 Here is some "some link text": "http://some.net/address"[] 

Overfull \hbox \(XXXpt too wide\) 
 []\T1/pcr/m/n/10 Links to files typeset in verbatim mode applies backtics:[] 

Overfull \hbox \(XXXpt too wide\) 
 []\T1/pcr/m/n/10 "`myfile.py`": "http://some.net/some/place/myfile.py".[] 

Overfull \hbox \(XXXpt too wide\) 
 []\T1/pcr/m/n/10 FIGURE: [relative/path/to/figurefile, width=500 frac=0.8] Her
e goes the caption which must be on a single line. label{some:fig:label}[] 

Overfull \hbox \(XXXpt too wide\) 
 []\T1/pcr/m/n/10 MOVIE: [relative/path/to/moviefile, width=500] Here goes the 
caption which must be on a single line.[] 

Overfull \hbox \(XXXpt too wide\) 
 []\T1/pcr/m/n/10 MOVIE: [http://www.youtube.com/watch?v=_O7iUiftbKU, width=420
 height=315] YouTube movie.[] 

Overfull \hbox \(XXXpt too wide\) 
 []\T1/pcr/m/n/10 MOVIE: [http://vimeo.com/55562330, width=500 height=278] Vime
o movie.[] 


ne 1371.

Overfull \hbox \(XXXpt too wide\) 
 []\T1/pcr/m/n/10 |----------------c--------|------------------c---------------
-----|[] 

Overfull \hbox \(XXXpt too wide\) 
 []\T1/pcr/m/n/10 |      Section type       |        Syntax                    
     |[] 

Overfull \hbox \(XXXpt too wide\) 
 []\T1/pcr/m/n/10 |----------------l--------|------------------l---------------
-----|[] 

Overfull \hbox \(XXXpt too wide\) 
 []\T1/pcr/m/n/10 | chapter                 | `========= Heading ========` (9 `
=`)  |[] 

Overfull \hbox \(XXXpt too wide\) 
 []\T1/pcr/m/n/10 | section                 | `======= Heading =======`    (7 `
=`)  |[] 

Overfull \hbox \(XXXpt too wide\) 
 []\T1/pcr/m/n/10 | subsection              | `===== Heading =====`        (5 `
=`)  |[] 

Overfull \hbox \(XXXpt too wide\) 
 []\T1/pcr/m/n/10 | subsubsection           | `=== Heading ===`            (3 `
=`)  |[] 

Overfull \hbox \(XXXpt too wide\) 
 []\T1/pcr/m/n/10 | paragraph               | `__Heading.__`               (2 `
_`)  |[] 

Overfull \hbox \(XXXpt too wide\) 
 []\T1/pcr/m/n/10 |------------------------------------------------------------
-----|[] 

Overfull \hbox \(XXXpt too wide\) 
 []\T1/pcr/m/n/10 Terminal> doconce csv2table mydata.csv > mydata_table.do.txt[
] 

Overfull \hbox \(XXXpt too wide\) 
\T1/ptm/m/n/10 sert a back-slash). Bib-li-og-ra-phy ci-ta-tions of-ten have \T1
/pcr/m/n/10 name \T1/ptm/m/n/10 on the form \T1/pcr/m/n/10 Author1_Author2_YYYY
\T1/ptm/m/n/10 ,

Overfull \hbox \(XXXpt too wide\) 
[]\T1/ptm/m/n/10 The bib-li-og-ra-phy is spec-i-fied by a line \T1/pcr/m/n/10 B
IBFILE: papers.pub\T1/ptm/m/n/10 , where \T1/pcr/m/n/10 papers.pub

Overfull \hbox \(XXXpt too wide\) 
 []\T1/pcr/m/n/10 ref[Section ref{subsec:ex}][in cite{testdoc:12}][a "section":

Overfull \hbox \(XXXpt too wide\) 
 []\T1/pcr/m/n/10 "A Document for Testing DocOnce": "testdoc.html" cite{testdoc
:12}],[] 

Overfull \hbox \(XXXpt too wide\) 
 []\T1/pcr/m/n/10 DocOnce version 1.5.7 (from /X/X/pyt
hon3.6/site-packages/DocOnce-1.5.7-py3.6.egg/doconce)[] 

Overfull \hbox \(XXXpt too wide\) 
 []\T1/pcr/m/n/10 commands: help format find subst replace remove spellcheck ap
ply_inline_edits capitalize change_encoding clean combine_images csv2table diff
 expand_commands expand_mako extract_exercises find_nonascii_chars fix_bibtex4p
ublish gitdiff grab grep guess_encoding gwiki_figsubst html2doconce html_colorb
ullets jupyterbook include_map insertdocstr ipynb2doconce latex2doconce latex_d
islikes latex_exercise_toc latex_footer latex_header latex_problems latin2html 
lightclean linkchecker list_fig_src_files list_labels makefile md2html md2latex
 old2new_format ptex2tex pygmentize ref_external remove_exercise_answers remove
_inline_comments replace_from_file slides_beamer slides_html slides_markdown sp
hinx_dir sphinxfix_localURLs split_html split_rst teamod[] 

Overfull \hbox \(XXXpt too wide\) 
 []\T1/pcr/m/n/10 doconce format html|latex|pdflatex|rst|sphinx|plain|gwiki|mwi
ki|[] 

Overfull \hbox \(XXXpt too wide\) 
 []\T1/pcr/m/n/10 # substitute a phrase by another using regular expressions (i
n this example -s is the re.DOTALL modifier, -m is the re.MULTILINE modifier, -
x is the re.VERBOSE modifier, --restore copies backup files back again)[] 

Overfull \hbox \(XXXpt too wide\) 
 []\T1/pcr/m/n/10 # replace a phrase by another literally (exact text substitut
ion)[] 

Overfull \hbox \(XXXpt too wide\) 
 []\T1/pcr/m/n/10 doconce replace_from_file file-with-from-to-replacements file
1 file2 ...[] 

Overfull \hbox \(XXXpt too wide\) 
 []\T1/pcr/m/n/10 # search for a (regular) expression in all .do.txt files in t
he current directory tree (useful when removing compilation errors)[] 

Overfull \hbox \(XXXpt too wide\) 
 []\T1/pcr/m/n/10 # print an overview of how various files are included in the 
root doc[] 

Overfull \hbox \(XXXpt too wide\) 
 []\T1/pcr/m/n/10 doconce expand_mako mako_code_file funcname file1 file2 ...[]
 

Overfull \hbox \(XXXpt too wide\) 
 []\T1/pcr/m/n/10 # replace all mako function calls by the `results of the call
s[] 

Overfull \hbox \(XXXpt too wide\) 
 []\T1/pcr/m/n/10 doconce sphinx_dir copyright='John Doe' title='Long title' \[
] 

Overfull \hbox \(XXXpt too wide\) 
 []        \T1/pcr/m/n/10 short_title="Short title" version=0.1 intersphinx \[]
 

Overfull \hbox \(XXXpt too wide\) 
 []\T1/pcr/m/n/10 # create a directory for the sphinx format (requires sphinx v
ersion >= 1.1)[] 

Overfull \hbox \(XXXpt too wide\) 
 []\T1/pcr/m/n/10 # split a sphinx/rst file into parts according to !split comm
ands[] 

Overfull \hbox \(XXXpt too wide\) 
 []\T1/pcr/m/n/10 # walk through a directory tree and insert doconce files as d
ocstrings in *.p.py files[] 

Overfull \hbox \(XXXpt too wide\) 
 []\T1/pcr/m/n/10 # remove all redundant files (keep source .do.txt and results
: .pdf, .html, sphinx- dirs, .mwiki, .ipynb, etc.)[] 

Overfull \hbox \(XXXpt too wide\) 
 []\T1/pcr/m/n/10 # split an html file into parts according to !split commands[
] 

Overfull \hbox \(XXXpt too wide\) 
 []\T1/pcr/m/n/10 # create LaTeX Beamer slides from a (doconce) latex/pdflatex 
file[] 

Overfull \hbox \(XXXpt too wide\) 
 []\T1/pcr/m/n/10 doconce slides_markdown complete_file.md remark --slide_style
=light[] 

Overfull \hbox \(XXXpt too wide\) 
 []\T1/pcr/m/n/10 doconce grab --from[-] from-text [--to[-] to-text] file > res
ult[] 

Overfull \hbox \(XXXpt too wide\) 
 []\T1/pcr/m/n/10 doconce remove --from[-] from-text [--to[-] to-text] file > r
esult[] 

Overfull \hbox \(XXXpt too wide\) 
 []\T1/pcr/m/n/10 doconce ptex2tex mydoc -DMINTED pycod=minted sys=Verbatim \[]
 

Overfull \hbox \(XXXpt too wide\) 
 []        \T1/pcr/m/n/10 dat=\begin{quote}\begin{verbatim};\end{verbatim}\end{
quote}[] 

Overfull \hbox \(XXXpt too wide\) 
 []\T1/pcr/m/n/10 # transform ptex2tex files (.p.tex) to ordinary latex file an
d manage the code environments[] 

Overfull \hbox \(XXXpt too wide\) 
 []\T1/pcr/m/n/10 doconce latex_problems mydoc.log [overfull-hbox-limit][] 

Overfull \hbox \(XXXpt too wide\) 
 []\T1/pcr/m/n/10 # list all figure files, movie files, and source code files n
eeded[] 

Overfull \hbox \(XXXpt too wide\) 
 []\T1/pcr/m/n/10 # list all labels in a document (for purposes of cleaning the
m up)[] 

Overfull \hbox \(XXXpt too wide\) 
 []\T1/pcr/m/n/10 # generate script for substituting generalized references[] 

Overfull \hbox \(XXXpt too wide\) 
 []\T1/pcr/m/n/10 # change headings from "This is a Heading" to "This is a head
ing"[] 

Overfull \hbox \(XXXpt too wide\) 
 []\T1/pcr/m/n/10 # translate a latex document to doconce (requires usually man
ual fixing)[] 

Overfull \hbox \(XXXpt too wide\) 
 []\T1/pcr/m/n/10 # check if there are problems with translating latex to docon
ce[] 

Overfull \hbox \(XXXpt too wide\) 
 []\T1/pcr/m/n/10 # typeset a doconce document with pygments (for pretty print 
of doconce itself)[] 

Overfull \hbox \(XXXpt too wide\) 
 []\T1/pcr/m/n/10 doconce makefile docname doconcefile [html sphinx pdflatex ..
.][] 

Overfull \hbox \(XXXpt too wide\) 
 []\T1/pcr/m/n/10 # generate a make.py script for translating a doconce file to
 various formats[] 

Overfull \hbox \(XXXpt too wide\) 
 []\T1/pcr/m/n/10 # find differences between two files (diffprog can be difflib
, diff, pdiff, latexdiff, kdiff3, diffuse, ...)[] 

Overfull \hbox \(XXXpt too wide\) 
 []\T1/pcr/m/n/10 # find differences between the last two Git versions of sever
al files[] 

Overfull \hbox \(XXXpt too wide\) 
 []\T1/pcr/m/n/10 # replace latex-1 (non-ascii) characters by html codes[] 

Overfull \hbox \(XXXpt too wide\) 
 []\T1/pcr/m/n/10 # fix common problems in bibtex files for publish import[] 

Overfull \hbox \(XXXpt too wide\) 
 []\T1/pcr/m/n/10 # insert a table of exercises in a latex file myfile.p.tex[] 

Overfull \hbox \(XXXpt too wide\) 
 []\T1/pcr/m/n/10 ===== Problem: Derive the Formula for the Area of an Ellipse 
=====[] 

Overfull \hbox \(XXXpt too wide\) 
 []\T1/pcr/m/n/10 Derive an expression for the area of an ellipse by integratin
g[] 

Overfull \hbox \(XXXpt too wide\) 
 []\T1/pcr/m/n/10 the area under a curve that defines half of the ellipse.[] 

Overfull \hbox \(XXXpt too wide\) 
 []\T1/pcr/m/n/10 ===== {Problem}: Derive the Formula for the Area of an Ellips
e =====[] 

Overfull \hbox \(XXXpt too wide\) 
 []\T1/pcr/m/n/10 ===== Exercise: Determine the Distance to the Moon =====[] 

Overfull \hbox \(XXXpt too wide\) 
 []\T1/pcr/m/n/10 Intro to this exercise. Questions are in subexercises below.[
] 

Overfull \hbox \(XXXpt too wide\) 
 []\T1/pcr/m/n/10 At the very end of the exercise it may be appropriate to summ
arize[] 

Overfull \hbox \(XXXpt too wide\) 
 []\T1/pcr/m/n/10 and give some perspectives. The text inside the `!bremarks` a
nd `!eremarks`[] 

Overfull \hbox \(XXXpt too wide\) 
 []\T1/pcr/m/n/10 directives is always typeset at the end of the exercise.[] 

Overfull \hbox \(XXXpt too wide\) 
\T1/ptm/m/n/10 DocOnce en-vi-ron-ments start with \T1/pcr/m/n/10 !benvirname \T
1/ptm/m/n/10 and end with \T1/pcr/m/n/10 !eenvirname\T1/ptm/m/n/10 , where

Overfull \hbox \(XXXpt too wide\) 

Overfull \hbox \(XXXpt too wide\) 
 []\T1/pcr/m/n/10 \multicolumn{1}{c}{time} & \multicolumn{1}{c}{velocity} & \mu
lticolumn{1}{c}{acceleration} \\[] 

Overfull \hbox \(XXXpt too wide\) 
[][][][][][] \T1/ptm/m/n/10 con-tains some il-lus-tra-tions on how to uti-lize 
\T1/pcr/m/n/10 mako \T1/ptm/m/n/10 (clone the GitHub

Overfull \hbox \(XXXpt too wide\) 
[]\T1/ptm/m/n/10 Excellent "Sphinx Tu-to-rial" by C. Reller: "[][][][][][]" 
[25] (./quickref.rst.aux)

LaTeX Warning: There were undefined references.


LaTeX Warning: Label(s) may have changed. Rerun to get cross-references right.

 )
(see the transcript file for additional information)
Output written on quickref.rst.dvi (XXX pages, ).
Transcript written on quickref.rst.log.
+ '[' 0 -ne 0 ']'
+ latex quickref.rst.tex
This is pdfTeX, Version 3.14159265-2.6-1.40.18 (TeX Live 2017/Debian) (preloaded format=latex)
 restricted \write18 enabled.
entering extended mode
(./quickref.rst.tex
LaTeX2e <2017-04-15>
Babel <3.18> and hyphenation patterns for 84 language(s) loaded.
(/usr/share/texlive/texmf-dist/tex/latex/base/article.cls
Document Class: article 2014/09/29 v1.4h Standard LaTeX document class

(/usr/share/texlive/texmf-dist/tex/latex/cmap/cmap.sty

Package cmap Warning: pdftex in DVI mode - exiting.

) 
(/usr/share/texlive/texmf-dist/tex/latex/base/fontenc.sty

(/usr/share/texlive/texmf-dist/tex/latex/base/inputenc.sty
(/usr/share/texlive/texmf-dist/tex/latex/base/utf8.def









(/usr/share/texlive/texmf-dist/tex/latex/psnfss/helvet.sty


(/usr/share/texlive/texmf-dist/tex/latex/hyperref/hyperref.sty
(/usr/share/texlive/texmf-dist/tex/generic/oberdiek/hobsub-hyperref.sty







(/usr/share/texlive/texmf-dist/tex/latex/hyperref/hdvips.def
(/usr/share/texlive/texmf-dist/tex/latex/hyperref/pdfmark.def

(/usr/share/texlive/texmf-dist/tex/latex/oberdiek/bookmark.sty

(./quickref.rst.aux) 
(/usr/share/texlive/texmf-dist/tex/latex/graphics/color.sty



(/usr/share/texlive/texmf-dist/tex/latex/hyperref/nameref.sty





 (./quickref.rst.toc

Overfull \hbox \(XXXpt too wide\) 
\T1/ptm/m/n/10 HTML. Other out-lets in-clude Google's \T1/pcr/m/n/10 blogger.co
m\T1/ptm/m/n/10 , Wikipedia/Wikibooks, IPython/Jupyter

Overfull \hbox \(XXXpt too wide\) 
 []\T1/pcr/m/n/10 AUTHOR: H. P. Langtangen at Center for Biomedical Computing, 
Simula Research Laboratory & Dept. of Informatics, Univ. of Oslo[] 

Overfull \hbox \(XXXpt too wide\) 
 []\T1/pcr/m/n/10 AUTHOR: Kaare Dump Email: dump@cyb.space.com at Segfault, Cyb
erspace Inc.[] 

Overfull \hbox \(XXXpt too wide\) 
 []\T1/pcr/m/n/10 name Email: somename@adr.net at institution1 & institution2[]

Overfull \hbox \(XXXpt too wide\) 
 []\T1/pcr/m/n/10 AUTHOR: name Email: somename@adr.net {copyright,2006-present}
 at inst1[] 

Underfull \hbox (badness 10000) 
[]|\T1/pcr/m/n/10 ======= Appendix: heading

Underfull \hbox (badness 1168) 
[]|\T1/pcr/m/n/10 ===== Appendix: heading ===== \T1/ptm/m/n/10 (5

Underfull \hbox (badness 1168) 
[]|\T1/pcr/m/n/10 ===== Exercise: heading ===== \T1/ptm/m/n/10 (5

Overfull \hbox \(XXXpt too wide\) 
\T1/ptm/m/n/10 Note that ab-stracts are rec-og-nized by start-ing with \T1/pcr/
m/n/10 __Abstract.__ \T1/ptm/m/n/10 or \T1/pcr/m/n/10 __Summary.__

Overfull \hbox \(XXXpt too wide\) 
\T1/ptm/m/it/10 sized words\T1/ptm/m/n/10 . Sim-i-larly, an un-der-score sur-ro
unds words that ap-pear in bold-face: \T1/pcr/m/n/10 _boldface_

Overfull \hbox \(XXXpt too wide\) 
 []\T1/pcr/m/n/10 This distance corresponds to 7.5~km, which is traveled in $7.
5/5$~s.[] 

Overfull \hbox \(XXXpt too wide\) 
 []\T1/pcr/m/n/10 The em-dash is used - without spaces - as alternative to hyph
en with[] 

Overfull \hbox \(XXXpt too wide\) 
 []\T1/pcr/m/n/10 *Premature optimization is the root of all evil.*--- Donald K
nuth.[] 

Overfull \hbox \(XXXpt too wide\) 
 []\T1/pcr/m/n/10 Note that sublists are consistently indented by one or more b
lanks as[] 

Overfull \hbox \(XXXpt too wide\) 
 []\T1/pcr/m/n/10 shown: bullets must exactly match and continuation lines must
 start[] 

Overfull \hbox \(XXXpt too wide\) 
 []\T1/pcr/m/n/10 Some running text. [hpl: There must be a space after the colo
n,[] 

Overfull \hbox \(XXXpt too wide\) 
 []\T1/pcr/m/n/10 doconce format html mydoc.do.txt --skip_inline_comments[] 

Overfull \hbox \(XXXpt too wide\) 
 []\T1/pcr/m/n/10 First consider a quantity $Q$. Without loss of generality, we
 assume[] 

Overfull \hbox \(XXXpt too wide\) 
 []\T1/pcr/m/n/10 $Q>0$. There are three, fundamental, basic property of $Q$.[]
 

Overfull \hbox \(XXXpt too wide\) 
 []\T1/pcr/m/n/10 we assume] $Q>0$. There are three[del: ,] fundamental[del: , 
basic][] 

Overfull \hbox \(XXXpt too wide\) 
 []\T1/pcr/m/n/10 [edit: property -> properties] of $Q$. [add: These are not[] 


Overfull \hbox \(XXXpt too wide\) 
\T1/ptm/m/n/10 three-color{red}{(\T1/ptm/b/n/10 del 5\T1/ptm/m/n/10 : ,}) fun-d
a-men-tal-color{red}{(\T1/ptm/b/n/10 del 6\T1/ptm/m/n/10 : , ba-sic}) \T1/ptm/b
/n/10 (**edit

Overfull \hbox \(XXXpt too wide\) 
 []\T1/pcr/m/n/10 # sphinx code-blocks: pycod=python cod=fortran cppcod=c++ sys
=console[] 

Overfull \hbox \(XXXpt too wide\) 
 []\T1/pcr/m/n/10 @@@CODE doconce_program.sh  fromto: doconce clean@^doconce sp
lit_rst[] 

Overfull \hbox \(XXXpt too wide\) 
 []\T1/pcr/m/n/10 @@@CODE doconce_program.sh  from-to: doconce clean@^doconce s
plit_rst[] 

Overfull \hbox \(XXXpt too wide\) 
 []\T1/pcr/m/n/10 @@@CODE doconce_program.sh  envir=shpro fromto: name=@[] 

Overfull \hbox \(XXXpt too wide\) 
 []\T1/pcr/m/n/10 \[ \frac{\partial\pmb{u}}{\partial t} + \pmb{u}\cdot\nabla\pm
b{u} = 0.\][] 

Overfull \hbox \(XXXpt too wide\) 
 []\T1/pcr/m/n/10 \[ \frac{\partial\pmb{u}}{\partial t} + \pmb{u}\cdot\nabla\pm
b{u} = 0.\][] 

Overfull \hbox \(XXXpt too wide\) 
[]\T1/ptm/m/n/10 If you want La-TeX math blocks to work with \T1/pcr/m/n/10 lat
ex\T1/ptm/m/n/10 , \T1/pcr/m/n/10 html\T1/ptm/m/n/10 , \T1/pcr/m/n/10 sphinx\T1
/ptm/m/n/10 ,

Overfull \hbox \(XXXpt too wide\) 
\T1/ptm/m/n/10 ments: \T1/pcr/m/n/10 \[ ... \]\T1/ptm/m/n/10 , \T1/pcr/m/n/10 e
quation*\T1/ptm/m/n/10 , \T1/pcr/m/n/10 equation\T1/ptm/m/n/10 , \T1/pcr/m/n/10
 align*\T1/ptm/m/n/10 , \T1/pcr/m/n/10 align\T1/ptm/m/n/10 .

Overfull \hbox \(XXXpt too wide\) 
\T1/pcr/m/n/10 alignat*\T1/ptm/m/n/10 , \T1/pcr/m/n/10 alignat\T1/ptm/m/n/10 . 
Other en-vi-ron-ments, such as \T1/pcr/m/n/10 split\T1/ptm/m/n/10 , \T1/pcr/m/n
/10 multiline\T1/ptm/m/n/10 ,

Overfull \hbox \(XXXpt too wide\) 
\T1/pcr/m/n/10 newcommands*.tex\T1/ptm/m/n/10 . Use \T1/pcr/m/n/10 \newcommands
 \T1/ptm/m/n/10 and not \T1/pcr/m/n/10 \def\T1/ptm/m/n/10 . Each

Package hyperref Warning: Ignoring empty anchor on .

Overfull \hbox \(XXXpt too wide\) 
[]  \T1/pcr/m/n/10 \includegraphics[width=0.55\linewidth]{figs/myfig.pdf}  

Overfull \hbox \(XXXpt too wide\) 
[]\T1/pcr/m/n/10 \multicolumn{1}{c}{$v_0$} & \multicolumn{1}{c}{$f_R(v_0)$}\\hl
ine  

Overfull \hbox \(XXXpt too wide\) 
 []\T1/pcr/m/n/10 Here is some "some link text": "http://some.net/address"[] 

Overfull \hbox \(XXXpt too wide\) 
 []\T1/pcr/m/n/10 Links to files typeset in verbatim mode applies backtics:[] 

Overfull \hbox \(XXXpt too wide\) 
 []\T1/pcr/m/n/10 "`myfile.py`": "http://some.net/some/place/myfile.py".[] 

Overfull \hbox \(XXXpt too wide\) 
 []\T1/pcr/m/n/10 FIGURE: [relative/path/to/figurefile, width=500 frac=0.8] Her
e goes the caption which must be on a single line. label{some:fig:label}[] 

Overfull \hbox \(XXXpt too wide\) 
 []\T1/pcr/m/n/10 MOVIE: [relative/path/to/moviefile, width=500] Here goes the 
caption which must be on a single line.[] 

Overfull \hbox \(XXXpt too wide\) 
 []\T1/pcr/m/n/10 MOVIE: [http://www.youtube.com/watch?v=_O7iUiftbKU, width=420
 height=315] YouTube movie.[] 

Overfull \hbox \(XXXpt too wide\) 
 []\T1/pcr/m/n/10 MOVIE: [http://vimeo.com/55562330, width=500 height=278] Vime
o movie.[] 

Overfull \hbox \(XXXpt too wide\) 
 []\T1/pcr/m/n/10 |----------------c--------|------------------c---------------
-----|[] 

Overfull \hbox \(XXXpt too wide\) 
 []\T1/pcr/m/n/10 |      Section type       |        Syntax                    
     |[] 

Overfull \hbox \(XXXpt too wide\) 
 []\T1/pcr/m/n/10 |----------------l--------|------------------l---------------
-----|[] 

Overfull \hbox \(XXXpt too wide\) 
 []\T1/pcr/m/n/10 | chapter                 | `========= Heading ========` (9 `
=`)  |[] 

Overfull \hbox \(XXXpt too wide\) 
 []\T1/pcr/m/n/10 | section                 | `======= Heading =======`    (7 `
=`)  |[] 

Overfull \hbox \(XXXpt too wide\) 
 []\T1/pcr/m/n/10 | subsection              | `===== Heading =====`        (5 `
=`)  |[] 

Overfull \hbox \(XXXpt too wide\) 
 []\T1/pcr/m/n/10 | subsubsection           | `=== Heading ===`            (3 `
=`)  |[] 

Overfull \hbox \(XXXpt too wide\) 
 []\T1/pcr/m/n/10 | paragraph               | `__Heading.__`               (2 `
_`)  |[] 

Overfull \hbox \(XXXpt too wide\) 
 []\T1/pcr/m/n/10 |------------------------------------------------------------
-----|[] 

Overfull \hbox \(XXXpt too wide\) 
 []\T1/pcr/m/n/10 Terminal> doconce csv2table mydata.csv > mydata_table.do.txt[
] 

Overfull \hbox \(XXXpt too wide\) 
\T1/ptm/m/n/10 sert a back-slash). Bib-li-og-ra-phy ci-ta-tions of-ten have \T1
/pcr/m/n/10 name \T1/ptm/m/n/10 on the form \T1/pcr/m/n/10 Author1_Author2_YYYY
\T1/ptm/m/n/10 ,

Overfull \hbox \(XXXpt too wide\) 
[]\T1/ptm/m/n/10 The bib-li-og-ra-phy is spec-i-fied by a line \T1/pcr/m/n/10 B
IBFILE: papers.pub\T1/ptm/m/n/10 , where \T1/pcr/m/n/10 papers.pub

Overfull \hbox \(XXXpt too wide\) 
 []\T1/pcr/m/n/10 ref[Section ref{subsec:ex}][in cite{testdoc:12}][a "section":

Overfull \hbox \(XXXpt too wide\) 
 []\T1/pcr/m/n/10 "A Document for Testing DocOnce": "testdoc.html" cite{testdoc
:12}],[] 

Overfull \hbox \(XXXpt too wide\) 
 []\T1/pcr/m/n/10 DocOnce version 1.5.7 (from /X/X/pyt
hon3.6/site-packages/DocOnce-1.5.7-py3.6.egg/doconce)[] 

Overfull \hbox \(XXXpt too wide\) 
 []\T1/pcr/m/n/10 commands: help format find subst replace remove spellcheck ap
ply_inline_edits capitalize change_encoding clean combine_images csv2table diff
 expand_commands expand_mako extract_exercises find_nonascii_chars fix_bibtex4p
ublish gitdiff grab grep guess_encoding gwiki_figsubst html2doconce html_colorb
ullets jupyterbook include_map insertdocstr ipynb2doconce latex2doconce latex_d
islikes latex_exercise_toc latex_footer latex_header latex_problems latin2html 
lightclean linkchecker list_fig_src_files list_labels makefile md2html md2latex
 old2new_format ptex2tex pygmentize ref_external remove_exercise_answers remove
_inline_comments replace_from_file slides_beamer slides_html slides_markdown sp
hinx_dir sphinxfix_localURLs split_html split_rst teamod[] 

Overfull \hbox \(XXXpt too wide\) 
 []\T1/pcr/m/n/10 doconce format html|latex|pdflatex|rst|sphinx|plain|gwiki|mwi
ki|[] 

Overfull \hbox \(XXXpt too wide\) 
 []\T1/pcr/m/n/10 # substitute a phrase by another using regular expressions (i
n this example -s is the re.DOTALL modifier, -m is the re.MULTILINE modifier, -
x is the re.VERBOSE modifier, --restore copies backup files back again)[] 

Overfull \hbox \(XXXpt too wide\) 
 []\T1/pcr/m/n/10 # replace a phrase by another literally (exact text substitut
ion)[] 

Overfull \hbox \(XXXpt too wide\) 
 []\T1/pcr/m/n/10 doconce replace_from_file file-with-from-to-replacements file
1 file2 ...[] 

Overfull \hbox \(XXXpt too wide\) 
 []\T1/pcr/m/n/10 # search for a (regular) expression in all .do.txt files in t
he current directory tree (useful when removing compilation errors)[] 

Overfull \hbox \(XXXpt too wide\) 
 []\T1/pcr/m/n/10 # print an overview of how various files are included in the 
root doc[] 

Overfull \hbox \(XXXpt too wide\) 
 []\T1/pcr/m/n/10 doconce expand_mako mako_code_file funcname file1 file2 ...[]
 

Overfull \hbox \(XXXpt too wide\) 
 []\T1/pcr/m/n/10 # replace all mako function calls by the `results of the call
s[] 

Overfull \hbox \(XXXpt too wide\) 
 []\T1/pcr/m/n/10 doconce sphinx_dir copyright='John Doe' title='Long title' \[
] 

Overfull \hbox \(XXXpt too wide\) 
 []        \T1/pcr/m/n/10 short_title="Short title" version=0.1 intersphinx \[]
 

Overfull \hbox \(XXXpt too wide\) 
 []\T1/pcr/m/n/10 # create a directory for the sphinx format (requires sphinx v
ersion >= 1.1)[] 

Overfull \hbox \(XXXpt too wide\) 
 []\T1/pcr/m/n/10 # split a sphinx/rst file into parts according to !split comm
ands[] 

Overfull \hbox \(XXXpt too wide\) 
 []\T1/pcr/m/n/10 # walk through a directory tree and insert doconce files as d
ocstrings in *.p.py files[] 

Overfull \hbox \(XXXpt too wide\) 
 []\T1/pcr/m/n/10 # remove all redundant files (keep source .do.txt and results
: .pdf, .html, sphinx- dirs, .mwiki, .ipynb, etc.)[] 

Overfull \hbox \(XXXpt too wide\) 
 []\T1/pcr/m/n/10 # split an html file into parts according to !split commands[
] 

Overfull \hbox \(XXXpt too wide\) 
 []\T1/pcr/m/n/10 # create LaTeX Beamer slides from a (doconce) latex/pdflatex 
file[] 

Overfull \hbox \(XXXpt too wide\) 
 []\T1/pcr/m/n/10 doconce slides_markdown complete_file.md remark --slide_style
=light[] 

Overfull \hbox \(XXXpt too wide\) 
 []\T1/pcr/m/n/10 doconce grab --from[-] from-text [--to[-] to-text] file > res
ult[] 

Overfull \hbox \(XXXpt too wide\) 
 []\T1/pcr/m/n/10 doconce remove --from[-] from-text [--to[-] to-text] file > r
esult[] 

Overfull \hbox \(XXXpt too wide\) 
 []\T1/pcr/m/n/10 doconce ptex2tex mydoc -DMINTED pycod=minted sys=Verbatim \[]
 

Overfull \hbox \(XXXpt too wide\) 
 []        \T1/pcr/m/n/10 dat=\begin{quote}\begin{verbatim};\end{verbatim}\end{
quote}[] 

Overfull \hbox \(XXXpt too wide\) 
 []\T1/pcr/m/n/10 # transform ptex2tex files (.p.tex) to ordinary latex file an
d manage the code environments[] 

Overfull \hbox \(XXXpt too wide\) 
 []\T1/pcr/m/n/10 doconce latex_problems mydoc.log [overfull-hbox-limit][] 

Overfull \hbox \(XXXpt too wide\) 
 []\T1/pcr/m/n/10 # list all figure files, movie files, and source code files n
eeded[] 

Overfull \hbox \(XXXpt too wide\) 
 []\T1/pcr/m/n/10 # list all labels in a document (for purposes of cleaning the
m up)[] 

Overfull \hbox \(XXXpt too wide\) 
 []\T1/pcr/m/n/10 # generate script for substituting generalized references[] 

Overfull \hbox \(XXXpt too wide\) 
 []\T1/pcr/m/n/10 # change headings from "This is a Heading" to "This is a head
ing"[] 

Overfull \hbox \(XXXpt too wide\) 
 []\T1/pcr/m/n/10 # translate a latex document to doconce (requires usually man
ual fixing)[] 

Overfull \hbox \(XXXpt too wide\) 
 []\T1/pcr/m/n/10 # check if there are problems with translating latex to docon
ce[] 

Overfull \hbox \(XXXpt too wide\) 
 []\T1/pcr/m/n/10 # typeset a doconce document with pygments (for pretty print 
of doconce itself)[] 

Overfull \hbox \(XXXpt too wide\) 
 []\T1/pcr/m/n/10 doconce makefile docname doconcefile [html sphinx pdflatex ..
.][] 

Overfull \hbox \(XXXpt too wide\) 
 []\T1/pcr/m/n/10 # generate a make.py script for translating a doconce file to
 various formats[] 

Overfull \hbox \(XXXpt too wide\) 
 []\T1/pcr/m/n/10 # find differences between two files (diffprog can be difflib
, diff, pdiff, latexdiff, kdiff3, diffuse, ...)[] 

Overfull \hbox \(XXXpt too wide\) 
 []\T1/pcr/m/n/10 # find differences between the last two Git versions of sever
al files[] 

Overfull \hbox \(XXXpt too wide\) 
 []\T1/pcr/m/n/10 # replace latex-1 (non-ascii) characters by html codes[] 

Overfull \hbox \(XXXpt too wide\) 
 []\T1/pcr/m/n/10 # fix common problems in bibtex files for publish import[] 

Overfull \hbox \(XXXpt too wide\) 
 []\T1/pcr/m/n/10 # insert a table of exercises in a latex file myfile.p.tex[] 

Overfull \hbox \(XXXpt too wide\) 
 []\T1/pcr/m/n/10 ===== Problem: Derive the Formula for the Area of an Ellipse 
=====[] 

Overfull \hbox \(XXXpt too wide\) 
 []\T1/pcr/m/n/10 Derive an expression for the area of an ellipse by integratin
g[] 

Overfull \hbox \(XXXpt too wide\) 
 []\T1/pcr/m/n/10 the area under a curve that defines half of the ellipse.[] 

Overfull \hbox \(XXXpt too wide\) 
 []\T1/pcr/m/n/10 ===== {Problem}: Derive the Formula for the Area of an Ellips
e =====[] 

Overfull \hbox \(XXXpt too wide\) 
 []\T1/pcr/m/n/10 ===== Exercise: Determine the Distance to the Moon =====[] 

Overfull \hbox \(XXXpt too wide\) 
 []\T1/pcr/m/n/10 Intro to this exercise. Questions are in subexercises below.[
] 

Overfull \hbox \(XXXpt too wide\) 
 []\T1/pcr/m/n/10 At the very end of the exercise it may be appropriate to summ
arize[] 

Overfull \hbox \(XXXpt too wide\) 
 []\T1/pcr/m/n/10 and give some perspectives. The text inside the `!bremarks` a
nd `!eremarks`[] 

Overfull \hbox \(XXXpt too wide\) 
 []\T1/pcr/m/n/10 directives is always typeset at the end of the exercise.[] 

Overfull \hbox \(XXXpt too wide\) 
\T1/ptm/m/n/10 DocOnce en-vi-ron-ments start with \T1/pcr/m/n/10 !benvirname \T
1/ptm/m/n/10 and end with \T1/pcr/m/n/10 !eenvirname\T1/ptm/m/n/10 , where

Overfull \hbox \(XXXpt too wide\) 

Overfull \hbox \(XXXpt too wide\) 
 []\T1/pcr/m/n/10 \multicolumn{1}{c}{time} & \multicolumn{1}{c}{velocity} & \mu
lticolumn{1}{c}{acceleration} \\[] 

Overfull \hbox \(XXXpt too wide\) 
[][][][][][] \T1/ptm/m/n/10 con-tains some il-lus-tra-tions on how to uti-lize 
\T1/pcr/m/n/10 mako \T1/ptm/m/n/10 (clone the GitHub

Overfull \hbox \(XXXpt too wide\) 
[]\T1/ptm/m/n/10 Excellent "Sphinx Tu-to-rial" by C. Reller: "[][][][][][]" 
[26] (./quickref.rst.aux)

LaTeX Warning: Label(s) may have changed. Rerun to get cross-references right.

 )
(see the transcript file for additional information)
Output written on quickref.rst.dvi (XXX pages, ).
Transcript written on quickref.rst.log.
+ dvipdf quickref.rst.dvi
+ system doconce format plain quickref --no_preprocess --no_abort
+ doconce format plain quickref --no_preprocess --no_abort
running mako on quickref.do.txt to make tmp_mako__quickref.do.txt
Translating doconce text in tmp_mako__quickref.do.txt to plain
copy complete file doconce_program.sh  (format: shpro)
*** made link to new HTML file movie_player1.html
    with code to display the movie 
    http://vimeo.com/55562330
output in quickref.txt
+ '[' 0 -ne 0 ']'
+ system doconce format gwiki quickref --no_preprocess --no_abort
+ doconce format gwiki quickref --no_preprocess --no_abort
running mako on quickref.do.txt to make tmp_mako__quickref.do.txt
Translating doconce text in tmp_mako__quickref.do.txt to gwiki
copy complete file doconce_program.sh  (format: shpro)
*** warning: footnotes are not supported for format gwiki
    footnotes will be left in the doconce syntax
*** made link to new HTML file movie_player1.html
    with code to display the movie 
    http://vimeo.com/55562330
output in quickref.gwiki
+ '[' 0 -ne 0 ']'
+ system doconce format mwiki quickref --no_preprocess --no_abort
+ doconce format mwiki quickref --no_preprocess --no_abort
running mako on quickref.do.txt to make tmp_mako__quickref.do.txt
Translating doconce text in tmp_mako__quickref.do.txt to mwiki
copy complete file doconce_program.sh  (format: shpro)
*** warning: footnotes are not supported for format mwiki
    footnotes will be left in the doconce syntax
*** made link to new HTML file movie_player1.html
    with code to display the movie 
    http://vimeo.com/55562330
output in quickref.mwiki
+ '[' 0 -ne 0 ']'
+ system doconce format cwiki quickref --no_preprocess --no_abort
+ doconce format cwiki quickref --no_preprocess --no_abort
running mako on quickref.do.txt to make tmp_mako__quickref.do.txt
Translating doconce text in tmp_mako__quickref.do.txt to cwiki
copy complete file doconce_program.sh  (format: shpro)
*** warning: footnotes are not supported for format cwiki
    footnotes will be left in the doconce syntax
*** made link to new HTML file movie_player1.html
    with code to display the movie 
    http://vimeo.com/55562330
output in quickref.cwiki
+ '[' 0 -ne 0 ']'
+ system doconce format st quickref --no_preprocess --no_abort
+ doconce format st quickref --no_preprocess --no_abort
running mako on quickref.do.txt to make tmp_mako__quickref.do.txt
Translating doconce text in tmp_mako__quickref.do.txt to st
copy complete file doconce_program.sh  (format: shpro)
*** warning: footnotes are not supported for format st
    footnotes will be left in the doconce syntax
*** made link to new HTML file movie_player1.html
    with code to display the movie 
    http://vimeo.com/55562330
output in quickref.st
+ '[' 0 -ne 0 ']'
+ system doconce format epytext quickref --no_preprocess --no_abort
+ doconce format epytext quickref --no_preprocess --no_abort
running mako on quickref.do.txt to make tmp_mako__quickref.do.txt
Translating doconce text in tmp_mako__quickref.do.txt to epytext
copy complete file doconce_program.sh  (format: shpro)
*** warning: footnotes are not supported for format epytext
    footnotes will be left in the doconce syntax
*** made link to new HTML file movie_player1.html
    with code to display the movie 
    http://vimeo.com/55562330
output in quickref.epytext
+ '[' 0 -ne 0 ']'
+ system doconce format pandoc quickref --no_preprocess --strict_markdown_output --github_md --no_abort
+ doconce format pandoc quickref --no_preprocess --strict_markdown_output --github_md --no_abort
running mako on quickref.do.txt to make tmp_mako__quickref.do.txt
Translating doconce text in tmp_mako__quickref.do.txt to pandoc
copy complete file doconce_program.sh  (format: shpro)
*** warning: footnotes are not supported for format pandoc
    footnotes will be left in the doconce syntax
output in quickref.md
+ '[' 0 -ne 0 ']'
+ rm -rf demo
+ mkdir demo
+ cp -r quickref.do.txt quickref.html quickref.p.tex quickref.tex quickref.pdf quickref.rst quickref.xml quickref.rst.html quickref.rst.tex quickref.rst.pdf quickref.gwiki quickref.mwiki quickref.cwiki quickref.txt quickref.epytext quickref.st quickref.md sphinx-rootdir/_build/html demo
cp: cannot stat 'quickref.p.tex': No such file or directory
+ cd demo
+ cat
+ echo

+ echo 'Go to the demo directory /X/X/demo and load index.html into a web browser.'
Go to the demo directory /X/X/demo and load index.html into a web browser.
+ cd ..
+ dest=../../pub/quickref
+ cp -r demo/html demo/quickref.pdf demo/quickref.html ../../pub/quickref
+ dest=../../../../doconce.wiki
+ cp -r demo/quickref.md ../../../../doconce.wiki